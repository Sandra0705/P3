\subsection[Egenskaber for \texorpdfstring{$e^{tA}$}{exp(tA)}]{Egenskaber for \bm{$e^{tA}$}}


\begin{minipage}\textwidth
\begin{thmx} \textbf{} \label{sæt:begyndelsesværdiproblemet_for_phi}%Ny sætning
\newline
Lad $s\in \R$ og $A\in \R^{n\times n}$. Lad $F(\Phi)=A\Phi$, hvorom det gælder, at $D_F \subseteq \R^{n\times n}$ er en åben ikke-tom delmængde og $F \in C^0(D_F,\R^{n\times n})$. $\E$ er defineret ud fra \autoref{def:løsning_autonomt_system}.
Der gælder for begyndelsesværdiproblemet
%
\begin{align}\label{eq:begyndelsesværdiproblemet_for_phi}
    \frac{d\Phi}{dt}= A\Phi, \quad \Phi(0) = \E(s)
\end{align}
%
at det har den entydig bestemte løsning
%
\begin{align}
    \Phi(t) = \E(t)\E(s) = \E(t+s)
\end{align}
%
hvor $t\in I$ og $\Phi\in C^1(D_F,\R^{n\times n})$
%
\end{thmx}
\end{minipage}

\begin{bev}\textbf{}\\
Lad $\Psi_1 = \E(t)\E(s)$. Det vil blive bevist, at $\Psi_1$ er en løsning til \eqref{eq:begyndelsesværdiproblemet_for_phi}.

$\Psi_1$ differentieres med hensyn til $t$.
%
%$\Psi_1$ indsættes i \eqref{eq:begyndelsesværdiproblemet_for_phi}.
\begin{align}\label{eq:e(t)e(s)_indsat_som_løsning}
    \frac{d\Psi_1}{dt} = 
    \frac{d\left(\E(t)\E(s)\right)}{dt}
\end{align}
%
Da $\E(s)$ ikke er afhængig af $t$, forbliver den konstant under differentiation med hensyn til $t$, og kan derfor rykkes udenfor parentesen i \eqref{eq:e(t)e(s)_indsat_som_løsning}. 
%
\begin{align*}
    \frac{d\left(\E(t)\E(s)\right)}{dt}&=\frac{d\E(t)}{dt}\E(s)
    \intertext{Det vides fra \autoref{def:løsning_autonomt_system}, at udtrykket kan omskrives til}
    \frac{d\E(t)}{dt}\E(s)&=A\E(t)\E(s) = A\Psi_1
\end{align*}

Herefter verificeres begyndelsesbetingelsen ved brug af \eqref{eq:løsning_e(t)=Ae(t)}.
% \begin{align*}
%     \E(0)\E(s)&=\E(s)\\
%     &\Updownarrow\\
%     I\E(s)&=\E(s)\\
%     &\Updownarrow\\
%     \E(s)&=\E(s)
% \end{align*}
\begin{align*}
        \Psi_1(0) &= \E(s)\\ 
        &= \E(0)\E(s)\\
        &=I_n\E(s)\\
        &=\E(s)
\end{align*}
Altså er begyndelsesbetingelsen opfyldt. \\
Hermed er det bevist, at $\Psi_1$ er en løsning til  \eqref{eq:begyndelsesværdiproblemet_for_phi}.

Lad $\Psi_2 = \E(t+s)$. Det vil blive bevist, at $\Psi_2$ også er en løsning til \eqref{eq:begyndelsesværdiproblemet_for_phi}.

%Dernæst vil det blive bevist, at $\Psi_2(t)=\E(t+s)$ er en løsning til \eqref{eq:begyndelsesværdiproblemet_for_phi}.

$\Psi_2$ differentieres med hensyn til $t$. Lad $\Tilde{t}=t+s$. 
\begin{align*}
    \frac{d\Psi_2}{dt} &= \frac{d\E(t+s)}{dt} = \frac{d\E(\Tilde{t})}{dt}
    \intertext{Kædereglen for differentiation anvendes, da $\Tilde{t}$ er en funktion af $t$. Ud fra \autoref{def:løsning_autonomt_system} kan ovenstående udskrives som}
    \frac{d\E(\Tilde{t})}{dt} &= A\E(\Tilde{t})\cdot\frac{d\Tilde{t}}{dt} \\
    %\intertext{$t+s$ bliver substitueret ind %igen.}
    %\frac{d\E(t+s)}{dt} &= %A\E(t+s)\cdot\frac{d(t+s)}{dt} \\
    \intertext{$t+s$ bliver substitueret ind igen og ved at differentiere det sidste led fås}
    \frac{d\E(t+s)}{dt} &= A\E(t+s)\cdot \frac{d(t+s)}{dt}\\
    &= A\E(t+s) \cdot 1\\
    &= A\E(t+s) = A\Psi_2
\end{align*}

Begyndelsesbetingelsen verificeres.
%
\begin{align*}
    \Psi_2(0)   &= \E(s)\\
                &= \E(0+s)\\
                &= \E(s)
\end{align*}
%
Altså er begyndelsesbetingelsen opfyldt. 

Hermed er det bevist, at $\Psi_2$ er en løsning til \eqref{eq:begyndelsesværdiproblemet_for_phi}.

Det vides ud fra \autoref{Afsnit:4.3}, at $\frac{d\Phi}{dt}$ har en entydig bestemt løsning, og derfor må det gælde for løsningen, at 
\begin{align*}
    \Phi(t) = \E(t)\E(s) = \E(t+s)
\end{align*}
\end{bev}

\begin{kor} \textbf{} \label{kor:egenskaber_til_begyndelsesværdiproblemet_for_phi}%Nyt korollar
\newline
Hvis $\E(t)\E(s) = \E(t+s)$, så gælder følgende
\begin{enumerate}
    \item $\E(t)\E(s)=\E(s)\E(t)$ for alle $t,s\in \R$
    \item $\E(t)$ er invertibel for alle $t\in \R$
    \item $\E(t)^{-1}= \E(-t)$ for alle $t\in \R$
\end{enumerate}
\end{kor}

\begin{bev}\textbf{} %Nyt bevis
\newline
\begin{itemize}
\item [] \textbf{Bevis for punkt 1.}\\
Fra \autoref{sæt:begyndelsesværdiproblemet_for_phi}, gælder
\begin{align*}
    \E(t)\E(s)&=\E(t+s)\\
    \intertext{Da $t$ og $s$ er reelle tal, er $(t+s)=(s+t)$.}
    \E(t+s)&=\E(s+t)\\
    &=\E(s)\E(t)
\end{align*}
Altså er $\E(t)\E(s)=\E(s)\E(t)$.
\item [] \textbf{Bevis for punkt 2 og 3.}\\
Lad $s=-t$, så gælder der, at
\begin{align*}
    \E(t)\E(-t) = \E\left(t+(-t)\right) = \E(0) = I_n\\
    \E(-t)\E(t) = \E\left((-t)+t\right) = \E(0) = I_n
\end{align*}
Da $\E(t)\E(-t) = \E(-t)\E(t) = I_n$, gælder det, at $\E(t)$ er invertibel for alle $t\in \R$, og herudfra er det også bevist, at $\E(t)^{-1} = \E(-t)$.
\end{itemize}
\end{bev}
