%$\E(t) = e^{tA}$ defineres til at være løsning til \autoref{eq:begyndelsesværdiproblemet_for_phi}

$e^{tA}$ defineres til at være den entydige løsning, $\E(t)$, til begyndelsesværdiproblemet \eqref{eq:løsning_e(t)=Ae(t)}. For $e^{tA}$ gælder følgende egenskaber.


%Hvis $\E(t) = e^{tA}$, så gælder følgende egenskaber.
\begin{thmx} \textbf{Egenskaber for} \label{sæt:egenskaber_for_eta} $\bm{e^{tA}}$\\
Lad $A\in\R^{n\times n}$, så har $e^{tA}$ følgende egenskaber. 
    \begin{enumerate}
        \item For alle $t\in \R$ gælder $\frac{d}{dt}e^{tA} = Ae^{tA}$
        \item Der gælder, at $e^{0A} = I_n$
        \item For alle $s,t\in \R$ gælder $e^{(s+t)A} = e^{sA} e^{tA}$
        \item For alle $t\in\R$ er $e^{tA}$ invertibel, og $(e^{tA})^{-1} = e^{-tA}$
    \end{enumerate}
\end{thmx}

Da $e^{tA}$ er defineret til at være $\E(t)$ forekommer ovenstående egenskaber ud fra \autoref{def:løsning_autonomt_system}, \autoref{sæt:begyndelsesværdiproblemet_for_phi} og \autoref{kor:egenskaber_til_begyndelsesværdiproblemet_for_phi}. Yderligere gælder følgende korollar for $e^{tA}$.

\begin{kor}\textbf{}\label{AB_kommuterer_kor}
\newline
    Hvis $A, B \in \R^{n\times n}$ kommuterer, altså $AB=BA$, så gælder det, at $e^{tA}B=Be^{tA}$ for alle $t \in \R$.
\end{kor}
%
\begin{bev} \textbf{}
\newline
Det antages først, at $B=A$. Fra \autoref{kor:egenskaber_til_begyndelsesværdiproblemet_for_phi} vides det, at $e^{sA}e^{tA} = e^{tA}e^{sA}$. Særtilfældet, hvor $B=A$, kan udtrykkes som følgende
%    
\begin{align*}
    e^{sA}e^{tA} &= e^{tA}e^{sA}\\
\intertext{Ovenstående differentieres med hensyn til $s$.}
    Ae^{sA}e^{tA} &= e^{tA}Ae^{sA}\\ 
\intertext{Lad $s=0$.}
    Ae^{0A}e^{tA} &= e^{tA}Ae^{0A}\\
    &\Updownarrow\\
    Ae^{tA} &= e^{tA}A
\intertext{Altså er særtilfældet, $B=A$, bevist.}
\end{align*}
%   
Antag derimod, at der eksisterer et vilkårligt $B\in\R^{n\times n}$, som opfylder $AB = BA$.
    %
\begin{align*}
\intertext{Lad $\Psi_1 = e^{tA}B$ og $\Psi_2 = Be^{tA}$. Det skal vises, at $\Psi_1$ og $\Psi_2$ løser samme begyndelsesværdiproblem. Først differentieres $\Psi_1$ og $\Psi_2$ med hensyn til $t$}
    \frac{d\Psi_1}{dt} = A e^{tA} B, &\qquad \quad \frac{d\Psi_2}{dt} = BA e^{tA} \\
\intertext{Da $AB$ kommuterer, kan $\Psi_1$ og $\Psi_2$ indsættes, hvilket giver følgende udtryk.}
    \frac{d\Psi_1}{dt} = A \Psi_1, &\qquad\quad\frac{d\Psi_2}{dt} = A\Psi_2 
\intertext{Derved er det vist, at $\Psi_1$ og $\Psi_2$ løser samme differentialligning. Begyndelsesbetingelsen bestemmes ved at sætte $t=0$ for $\Psi_1$ og $\Psi_2$}
    \Psi_1(0) &= e^{0A} B = B\\
    \Psi_2(0) &= B e^{0A} = B
\end{align*} 
Det er derved vist, at $\Psi_1$ og $\Psi_2$ er løsninger til samme begyndelsesværdiproblem. Ud fra eksistens- og entydighedssætningen (se \autoref{sæt:eksistens_og_entydighed}), må $\Psi_1$ = $\Psi_2$, hvorved \autoref{AB_kommuterer_kor} er bevist.
\end{bev}

