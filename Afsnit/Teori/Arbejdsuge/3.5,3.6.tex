\subsection{Sætning 3.5 og opgave 3.6}
\textbf{Vi sætter $\E(t) = e^tA$}
\begin{thmx} \textbf{Egenskaber for} $\bm{e^{tA}}$\\
For $A\in\R^{n\times n}$ har $e^{tA}$ følgende egenskaber. 
    \begin{enumerate}
        \item Der gælder at $\frac{d}{dt}e^{tA} = Ae^{tA}$
        \item Der gælder at $e^{0A} = I$
        \item For alle $s,t\in \R$ gælder $e^{(s+t)A} = e^{sA} e^{tA}$
        \item For alle $t\in\R$ er $e^{tA}$ invertibel, og $(e^{tA})^{-1} = e^{-tA}$
    \end{enumerate}
\end{thmx}
\textbf{Sætningen gælder ud fra ovenstående}

\begin{kor}\textbf{Titel}\label{AB_kommuterer_kor}
\newline
    Hvis $A, B \in \R^{n\times n}$ kommuterer, dvs. $AB=BA$. Så gælder, at $e^{tA}B=Be^{tA}$ for alle $t \in \R$.
\end{kor}

\begin{bev}
    \begin{enumerate}\textbf{} \newline
        \item Det antages først, at $B=A$. Fra \autoref{kor:egenskaber_til_begyndelsesværdiproblemet_for_phi} vides det, at $e^{sA}e^{tA} = e^{tA}e^{sA}$. Særtilfældet hvor $A=B$ kan vises som følgende
    %    
        \begin{align*}
            e^{sA}e^{tA} &= e^{tA}e^{sA}\\ % .
            \intertext{Der differentieres med hensyn til $s$.}
            Ae^{sA}e^{tA} &= e^{tA}Ae^{sA}\\ % .
            \intertext{Lad $s=0$}
            Ae^{0A}e^{tA} &= e^{tA}Ae^{0A}\\
            Ae^{tA} &= e^{tA}A
            \intertext{hvilket beviser særtilfældet $A=B$.}
        \end{align*}
     %   
        \item Antag derimod, at der eksisterer et vilkårligt $B\in\R^{n\times n}$, som opfylder $AB = BA$.
    %
        \begin{align*}
            \intertext{Lad $\Psi_1 = e^{tA}B$ og lad $\Psi_2 = Be^{tA}$ det skal vises, at $\Psi_1$ og $\Psi_2$ løser samme differentialligning med samme begyndelsesbetingelse. Først differentieres $\Psi_1$ og $\Psi_2$}
            \frac{d\Psi_1}{dt} = A e^{tA} B, &\qquad \quad \frac{d\Psi_2}{dt} = BA e^{tA} \\
            \intertext{Ved at indsætte $\Psi_1$ og $\Psi_2$ fås}
            \frac{d\Psi_1}{dt} = A \Psi_1, &\qquad \quad \frac{d\Psi_2}{dt} = A\Psi_2 
            \intertext{Derved er det vist, at $\Psi_1$ og $\Psi_2$ løser samme differentialligning. Begyndelsesbetingelsesn bestemmed ved at sætte $t=0$ for $\Psi_1$ og $\Psi_2$}
            \Psi_1(0) = e^{0A} B, &\qquad \quad \Psi_2(0) = B e^{0A} \\ 
            \Psi_1(0) = B, &\qquad \quad \Psi_2(0) = B \\
            \intertext{Det er derved vist, at $\Psi_1$ og $\Psi_2$ er løsninger til samme begyndelsesværdiproblem. Ud fra eksistens og entydighedsætningen, \autoref{sæt:eksistens_og_entydighed}, må $\Psi_1$ = $\Psi_2$}
        \end{align*} 
    \end{enumerate}
\end{bev}