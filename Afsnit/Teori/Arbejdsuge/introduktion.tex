I det forrige kapitel blev autonome lineære differentialligninger, samt tilhørende løsninger gennemgået. Dette skal kunne generaliseres til autonome lineære systemer, hvor der indgår $n$ autonome lineære differentialligninger.  

Det homogene autonome lineære system af første ordens differentialligninger kan udtrykkes som følgende begyndelsesværdiproblem
%
\begin{align} \label{eq:begyndelsesværdiproblem_basic}
    \frac{d\textbf{x}}{dt}=A\textbf{x}, \quad \textbf{x}(t_0)=\textbf{x}_0
\end{align}
%
hvor $A \in \R^{n \times n}$ og $\textbf{x}: I \to \R^n$. Den entydigt bestemte løsning vil da være $\textbf{x}(t)=e^{(t-t_0)A}\textbf{x}_0$ jævnfør \autoref{sæt:løsning_til_homogen_system_af_første_ordens_differentialligninger}. Denne sætning vil blive bevist senere i projektet. Udtrykket indeholder en matriceeksponential $e^{tA}$, hvor $e^{tA}: I\times\R^{n\times n} \to \R^{n \times n}$. Denne matriceeksponential defineres således, at $\textbf{x}(t)$ er løsningen til \eqref{eq:begyndelsesværdiproblem_basic}.
%


%For at dette er gældende, skal der for ethvert $A$ kunne defineres en matrice $e^{tA}$, som giver løsningen til $\eqref{eq:begyndelsesværdiproblem_basic}$.

Lad $\Phi(t)$ betegne en løsning til \eqref{eq:begyndelsesværdiproblem_basic}, hvor $\Phi(t): I\to \R^{n \times n}$. Begyndelseværdiproblemet for følgende differentialligning opstilles.
%
\begin{align*}
    \frac{d\Phi}{dt}= F(t,\Phi), \quad \Phi(0)=I_n
\end{align*}
%
Under antagelsen om, at systemet er lineært, 
%Eftersom der arbejdes med lineære differentialligninger, 
kan systemet omskrives til følgende, hvor $A \in \R^{n \times n}$.
%
%, og lad $\Phi(t)$ betegne en løsning til \eqref{eq:begyndelsesværdiproblem_basic}. Begyndelsesværdiproblemet kan da opstilles som
%
\begin{align}\label{eq:begyndelsesværdiproblemet_med_phi}
  \frac{d\Phi}{dt}=A\Phi, \quad \Phi(0)=I_n
\end{align}
%
hvor $\Phi(t)$ er en differentiabel funktion med værdier i $\R^{n \times n}$, og $I_n$ betegner identitetesmatricen for $\R^{n\times n}$. Da $\Phi(t)$ er en matrice, kan den udtrykkes som følgende
\begin{align*}
    \Phi(t)=\begin{bmatrix}\mathbf{s}_1 & \mathbf{s}_2 & \cdots & \mathbf{s}_n\end{bmatrix}
\end{align*}
%
hvor $s_j$ betegner den $j$'te søjle, $j=1, 2, \cdots, n$. Differentialligningen i \eqref{eq:begyndelsesværdiproblemet_med_phi} kan omskrives til
\begin{align}
    \frac{d}{dt}\begin{bmatrix}\mathbf{s}_1 & \mathbf{s}_2 & \cdots & \mathbf{s}_n\end{bmatrix}=A\begin{bmatrix}\mathbf{s}_1 & \mathbf{s}_2 & \cdots & \mathbf{s}_n\end{bmatrix} \label{eq:omskrivning_begyndelsesværdiproblem}
\end{align}
%
Begyndelsesværdiproblemet er ud fra \eqref{eq:omskrivning_begyndelsesværdiproblem} ækvivalent med systemet
\begin{align*}
    \frac{d \mathbf{s}_1}{dt}&=A\mathbf{s}_1, \quad \mathbf{s}_1(0)=\mathbf{e}_1\\
    \frac{d \mathbf{s}_2}{dt}&=A\mathbf{s}_2, \quad \mathbf{s}_2(0)=\mathbf{e}_2\\
    &\vdots \quad\quad\quad\quad\quad\quad \vdots\\
    \frac{d \mathbf{s}_n}{dt}&=A\mathbf{s}_n, \quad \mathbf{s}_n(0)=\mathbf{e}_n
\end{align*}
%
hvor begyndelsesbetingelsen $s_j(0)$ er givet ved enhedsvektorerne i $\R^n$, hvilket er ensbetydende med $I_{n}=\begin{bmatrix}\textbf{e}_1 & \textbf{e}_2 & \cdots & \textbf{e}_n\end{bmatrix}$.

Herefter ønskes det at vise, at $\Phi(t)$ er en entydig løsning til \eqref{eq:begyndelsesværdiproblem_basic}. Dette gøres ved at anvende Lipschitz-betingelsen fra eksistens- og entydighedssætningen (se \autoref{sæt:eksistens_og_entydighed}).

Lad derfor $D_F \subseteq \R \times \R^{n \times n}$ være en åben og ikke-tom delmængde, og $F: D_F \to \R^{n \times n}$. Lad $\Phi$ og $\Tilde{\Phi}$ være to løsninger til \eqref{eq:begyndelsesværdiproblem_basic}, samt $F(t,\Phi)=A\Phi$ og $F(t,\Tilde{\Phi})=A\Tilde{\Phi}$.

Venstresiden af Lipschitz-betingelsen opskrives for $F(t, \Phi)$ og $F(t, \Tilde{\Phi})$ 
\begin{align*}
    \norm{F(t,\Phi)-F(t,\Tilde{\Phi})}_F 
    &= \norm{A\Phi-A\Tilde{\Phi}}_F \\ 
    &=\norm{A(\Phi-\Tilde{\Phi})}_F 
\intertext{\autoref{sæt:frobenius_egenskab_1} anvendes.}
    \norm{A(\Phi-\Tilde{\Phi})}_F &\leq \norm{A}_F\norm{\Phi-\Tilde{\Phi}}_F
\end{align*}
%
Da $\norm{A}_F \in \R_{\geq 0}$, kan der altid vælges et $L \in \R_{>0}$, hvor $\norm{A}_F \leq L$. Lipschitz-betingelsen er da opfyldt, hvoraf der vil gælde, at
%Hvis $\norm{A}_F \leq L$ er Lipschitz-betingelsen opfyldt, da der vil gælde, at %
\begin{align*}
    \norm{F(t,\Phi)-F(t,\Tilde{\Phi})}_F \leq L\norm{\Phi-\Tilde{\Phi}}_F
\end{align*}
%
Da Lipschitz-betingelsen i variablerne $t$ og $\Phi$ er opfyldt, eksisterer der, jævnfør eksistens- og entydighedssætningen (\autoref{sæt:eksistens_og_entydighed}), en entydig løsning til \eqref{eq:begyndelsesværdiproblem_basic}.

Denne løsning vil fremadrettet blive betegnet som $\E(t)$, og bliver defineret som følgende.

\begin{defn}\textbf{Løsning til et autonomt lineært system af differentialligninger} \label{def:løsning_autonomt_system}
\newline
Lad $A\in\R^{n\times n}$. Lad $F(\E)=A\E$, hvorom det gælder, at $D_F\subseteq \R^{n\times n}$ er en åben ikke-tom delmængde og $F\in C^0(D_F,\R^{n\times n})$. Så kan følgende begyndelsesværdiproblem opstilles
%Da er $\E(t)$ den entydige løsning til begyndelsesværdiproblemet
%
\begin{align}
    \frac{d\E}{dt}=A\E, \quad \E(0)=I_n \label{eq:løsning_e(t)=Ae(t)}
\end{align}
%
hvor $t \in I$ og $\E\in C^1(I,D_F)$. 
\end{defn}










