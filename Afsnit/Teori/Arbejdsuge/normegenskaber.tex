I dette kapitel vil termer, resultater og tilhørende egenskaber, som er centrale for gennemarbejdningen af de fremtidige kapitler blive gennemgået.

\section{Indre produkt}
Det \textit{euklidiske indre produkt} anvendes til at bevise relevante resultater, såsom egenskaberne tilhørende norm, Cauchy-Schwarz' ulighed med mere. Det euklidiske indre produkt er defineret ved følgende.
%
\begin{defn}\textbf{Euklidisk indre produkt}\label{indre_produkt_Eurlidisk}
\newline
    Lad $\textbf{u}, \textbf{v} \in V$, hvor $V$ er et vektorrum i $\R^n$. Så vil det euklidiske indre produkt af disse være givet ved
    \begin{equation*}
     {\langle \textbf{u},\textbf{v}\rangle} = \sum_{k=1}^{n}\textbf{$u_k$}\textbf{$v_k$}.
    \end{equation*}
\end{defn}

Fremadrettet vil det euklidiske indre produkt blive refereret til som det \textit{indre produkt}.

Hvis der eksisterer et $n$-dimensionelt vektorrum, $V$, hvor man kan bestemme det indre produkt mellem alle vektorer, kaldes vektorrummet for et \textit{indre produktrum}.
%
\begin{defn}\textbf{Indre produktrum}\label{indre_produktrum}
\newline
    Et indre produktrum er et vektorrum, $V$ sammen med et indre produkt på $V$. 
\end{defn}
%
Et indre produktrum har en række egenskaber.\\

\begin{defn}\textbf{Indre produkt}\label{indre_produkt_defn}
    \newline
    Et indre produkt på et vektorrum $V \subseteq \R^n$ er en funktion ${\langle \cdot,\cdot \rangle}: V \times V \to \R$, der har følgende egenskaber:\\
    \begin{itemize}
        \item[1.]{\makebox[5.2cm]{${\langle \textbf{u},\textbf{v} \rangle} = {\langle \textbf{v},\textbf{u} \rangle}$ \hfill}} \quad (Symmetri)
         \item[2.]{\makebox[5.2cm]{${\langle \textbf{u} + \textbf{v},\textbf{w} \rangle}  = {\langle \textbf{u},\textbf{w} \rangle} + {\langle \textbf{v},\textbf{w} \rangle}$ \hfill}} \quad (Additivitet i første indgang)
         \item[3.]{\makebox[5.2cm]{${\langle c \cdot \textbf{u},\textbf{v} \rangle} = c \cdot {\langle \textbf{u},\textbf{v} \rangle}$ \hfill}} \quad (Homogenitet i første indgang)
         \item[4.]{\makebox[5.2cm]{${\langle \textbf{u},\textbf{u} \rangle} \geq 0$ for alle $\textbf{u} \in V$ \hfill}} \quad (Positivitet)
         \item[5.]{\makebox[5.2cm]{${\langle \textbf{u},\textbf{u} \rangle} = \textbf{0} \Leftrightarrow \textbf{u}=\textbf{0}$ \hfill}} \quad (Definithed)
    \end{itemize}
    hvor $c \in \R$.
\end{defn}
%I dette projekt opereres der kun i de reelle tal, reelle vektorer samt reelle matricer, 

%I dette projekt opereres der kun i $\R^{m \times n}$, og derfor er ovenstående definitioner simplificeret. 
Et indre produkt kan godt forekomme i et komplekst vektorrum, hvori det indre produkt kan afbilde hen i et komplekst vektorrum. Det indre produkt over et komplekst vektorrum anvendes ikke i dette projekt, og derfor er ovenstående definition simplificeret over reelle vektorrum.

\section{Norm}
En \textit{norm} er defineret på følgende måde.
\begin{defn}\textbf{Norm}\label{def:norm} %Ny definition
\newline
Lad $\textbf{u} \in \R^n$. Normen af \textbf{u} er givet ved 
\begin{align*}
    \|\mathbf{u}\| = \sqrt{\sum^n_{k=1}|u_k|^2}
\end{align*}
\end{defn}

Ud fra \autoref{indre_produkt_Eurlidisk} og \autoref{def:norm} bemærkes det, at
\begin{align}\label{normen_i_anden_er_lig_kvadratroden_af_indreprodukt}
    \norm{\textbf{u}}^2 = \langle \textbf{u},\textbf{u}\rangle
\end{align}

Et vigtigt resultat som følger af definitionen af det indre produkt og norm er Cauchy-Schwarz' ulighed. 
Følgende sætning og det tilhørende bevis er baseret på \cite[s. 172]{"LAMA"}.

\begin{thmx}\label{sæt:cauchy-schwarz} \textbf{Cauchy-Schwarz' ulighed} 
\newline
Lad $\textbf{u},\textbf{v} \in \R^n$. Da gælder følgende ulighed.
\begin{align}\label{eq:cauchy-schwarz_ulighed}
    \abs{\langle \textbf{u},\textbf{v} \rangle} \leq \norm{\textbf{u}}\norm{\textbf{v}} 
\end{align}
Uligheden vil være en lighed, hvis og kun hvis vektorerne \textbf{u} og \textbf{v} er lineært afhængige.
\end{thmx}
%
\begin{bev}\textbf{}
\newline
Først bevises det trivielle tilfælde, hvor $\textbf{u}=\textbf{0}$ eller $\textbf{v}=\textbf{0}$. Her vil både højre- og venstresiden i \eqref{eq:cauchy-schwarz_ulighed} være lig nul $0$, hvoraf uligheden er opfyldt.

Dernæst bevises Cauchy-Schwarz' ulighed (\autoref{sæt:cauchy-schwarz}) for $\textbf{v}\neq\textbf{0}$. Lad $\textbf{v} \neq \textbf{0}$ og lad en ortogonal dekomposition (se bilag, \autoref{sæt:ortogonal_dekomposition}, side \pageref{sæt:ortogonal_dekomposition}) være givet ved
\begin{align*}
    \textbf{u}&=\frac{\langle\textbf{u},\textbf{v}\rangle}{\norm{\textbf{v}}^2}\textbf{v}+\textbf{w} \\
\intertext{hvor $\textbf{w}$ er ortogonal på $\textbf{v}$. Ud fra Pythagoras' sætning (se bilag, \autoref{sæt:pythagoras_sætning}, side \pageref{sæt:pythagoras_sætning}) gælder følgende}
    \norm{\textbf{u}}^2
    &=\norm{\frac{\langle\textbf{u},\textbf{v}\rangle}{\norm{\textbf{v}}^2}\textbf{v}}^2+\norm{\textbf{w}}^2 \\
    &= \left\langle {\frac{\langle\textbf{u},\textbf{v}\rangle}{\norm{\textbf{v}}^2}\textbf{v}} , {\frac{\langle\textbf{u},\textbf{v}\rangle}{\norm{\textbf{v}}^2}\textbf{v}} \right\rangle + \norm{\textbf{w}}^2 \\
\intertext{Symmetri og homogenitet i første indgang fra \autoref{indre_produkt_defn} anvendes til at isolere $\displaystyle\frac{\langle\textbf{u},\textbf{v}\rangle}{\norm{\textbf{v}}^2}$, hvorefter udtrykket reduceres.}
    \norm{\textbf{u}}^2 &=
    \left\langle {\frac{\langle\textbf{u},\textbf{v}\rangle}{\norm{\textbf{v}}^2}\textbf{v}} , {\frac{\langle\textbf{u},\textbf{v}\rangle}{\norm{\textbf{v}}^2}\textbf{v}} \right\rangle + \norm{\textbf{w}}^2 \\
    &=  \left(\frac{\langle\textbf{u},\textbf{v}\rangle}{\norm{\textbf{v}}^2}\right)^2\langle\textbf{v},\textbf{v}\rangle + \norm{\textbf{w}}^2\\
    &= {\frac{\langle\textbf{u},\textbf{v}\rangle{\langle\textbf{u},\textbf{v}\rangle}}{\norm{\textbf{v}}^4}\norm{\textbf{v}}^2} + \norm{\textbf{w}}^2 \\
    &= \frac{\abs{\langle\textbf{u},\textbf{v}\rangle}^2}{\norm{\textbf{v}}^2}+\norm{\textbf{w}}^2 \\
    &\geq \frac{\abs{\langle\textbf{u},\textbf{v}\rangle}^2}{\norm{\textbf{v}}^2}\\
\intertext{Herefter multipliceres begge sider af uligheden med $\norm{\textbf{v}}^2$.}
    \norm{\textbf{u}}^2 \norm{\textbf{v}}^2 &\geq
    \frac{\abs{\langle\textbf{u},\textbf{v}\rangle}^2}{\norm{\textbf{v}}^2}\norm{\textbf{v}}^2 \\
    &= \abs{\langle\textbf{u},\textbf{v}\rangle}^2
\intertext{Ved at tage kvadratroden på begge sider fås}
    \sqrt{\norm{\textbf{u}}^2\norm{\textbf{v}}^2} &\geq \sqrt{\abs{\langle\textbf{u},\textbf{v}\rangle}^2}\\
    &\Updownarrow \\
    \norm{\textbf{u}}\norm{\textbf{v}} &\geq \abs{\langle\textbf{u},\textbf{v}\rangle}
\end{align*}
Herved er den ønskede ulighed bevist. Det bemærkes ud fra \autoref{sæt:ortogonal_dekomposition}, at uligheden er en lighed, hvis og kun hvis $\textbf{w}=\textbf{0}$, da $\textbf{u} = c\textbf{v} + \textbf{0} = c\textbf{v}$.
I dette tilfælde vil $\textbf{u}$ og $\textbf{v}$ være lineært afhængige. Dermed er \eqref{eq:cauchy-schwarz_ulighed} bevist.
\end{bev}

\begin{thmx}\textbf{Regneregler for norm}\label{sæt:norm2} %Ny definition
\newline
Lad $\mathbf{u}, \mathbf{v} \in \R^n$. En funktion $\norm{\cdot}: V \to \R$ kaldes for en norm, hvis følgende betingelser er opfyldt:
\begin{enumerate}
    \item $\|\mathbf{u}\|\geq 0$
    \item $\|\mathbf{u}\|=0 \Leftrightarrow \mathbf{u=0}$
    \item $\|c\mathbf{u}\|=|c| \|\mathbf{u}\|, \quad c\in \R$
    \item $\|\mathbf{u}+\mathbf{v}\|\leq \|\mathbf{u}\|+\|\mathbf{v}\|$
\end{enumerate}
\end{thmx}
%
\begin{bev} \textbf{} %Nyt bevis
\newline
Beviserne for \autoref{sæt:norm2} anvender egenskaberne for indre produkt, \autoref{indre_produkt_defn}.\\
Lad $\mathbf{u}, \mathbf{v} \in \R^n$, samt lad $c\in \R$ være en vilkårlig skalar.

\begin{enumerate}
    \item[] \textbf{Bevis for punkt 1}.\\
        Ud fra \autoref{def:norm} kan normen skrives som følgende:
        \begin{align*}
            \|\mathbf{u}\| = \sqrt{\sum^n_{k=1}|u_k|^2}
        \end{align*}
        Summen af reelle kvadrerede ikke-negative tal giver et reelt ikke-negativ tal. Kvadratroden af dette ikke-negative tal giver også et ikke-negativt tal, hvilket medfører, at normen, $\|\textbf{u}\|$, er et ikke-negativt tal. Altså
        \begin{align*}
            \| \mathbf{u} \| \geq 0
        \end{align*}
        Ovenstående ulighed er kun en lighed i det specifikke tilfælde, hvor $\textbf{u} = \textbf{0}$. Dette bevises i næste punkt.
    \item[] \textbf{Bevis for punkt 2}.\\ 
        Ud fra definitionen af en norm, \autoref{def:norm}, skal alle elementer, $u_k$, i summen være nul for, at højresiden i udtrykket giver 0. Da alle elementer $u_k$ er lig nul, skal $\textbf{u}$ være nulvektoren. Altså gælder $$\norm{\textbf{u}}=0 \quad \Leftrightarrow \quad \mathbf{u}=\textbf{0}$$
   \item[] \textbf{Bevis for punkt 3}.\\
        Da $c \in \R$ gælder der, at 
        \begin{align*}
            \| c\mathbf{u}\|^2&=\langle c\textbf{u}, c\textbf{u} \rangle\\
            &= c\langle \textbf{u}, c\textbf{u} \rangle\\
            &= c\cdot c\langle \textbf{u},\textbf{u} \rangle\\
            &= c^2\|\textbf{u}\|^2
            \intertext{Ved at tage kvadratroden på begge sider fås}
           %\sqrt{ \| c\mathbf{u}\|^2}&= \sqrt{|c|^2\|\textbf{u}\|^2}\\
           \|c\textbf{u}\|&=|c|\|\textbf{u}\|
        \end{align*}
    \item[] \textbf{Bevis for punkt 4}.\\
        Der gælder, at
        \begin{align*}
            \|\textbf{u}+\textbf{v}\|^2 &= \langle \textbf{u}+\textbf{v}, \textbf{u} + \textbf{v} \rangle\\
            &= \langle \textbf{u},\textbf{u} \rangle + \langle \textbf{v}, \textbf{v} \rangle + \langle \textbf{u}, \textbf{v} \rangle + \langle \textbf{v}, \textbf{u} \rangle\\
            &=\langle \textbf{u},\textbf{u} \rangle + \langle \textbf{v}, \textbf{v} \rangle + 2\langle \textbf{u}, \textbf{v} \rangle\\
            \intertext{Ved at tage absolutværdien af $\langle \textbf{u}, \textbf{v}\rangle$ fås følgende ulighed}
            \langle \textbf{u},\textbf{u} \rangle + \langle \textbf{v}, \textbf{v} \rangle + 2\langle \textbf{u}, \textbf{v} \rangle &\leq \|\textbf{u}\|^2+ \|\textbf{v}\|^2 + 2|\langle \textbf{u}, \textbf{v}\rangle|\\
            \intertext{Ud fra Cauchy-Schwarz' ulighed gælder følgende}
             \norm{\textbf{u}+ \textbf{v}}^2 &\leq \|\textbf{u}\|^2+ \|\textbf{v}\|^2 + 2|\langle \textbf{u}, \textbf{v}\rangle|\\ &\leq \|\textbf{u}\|^2+ \|\textbf{v}\|^2 + 2\|\textbf{u}\|\|\textbf{v}\|\\
            &=\left(\|\textbf{u}\|+\|\textbf{v}\|\right)^2\\
            \intertext{Ved at tage kvadratroden på begge sider gælder følgende}
           \sqrt{ \|\textbf{u}+\textbf{v}\|^2} &\leq \sqrt{\left(\|\textbf{u}\|+\|\textbf{v}\|\right)^2}\\
           &\Updownarrow\\
           \|\textbf{u}+\textbf{v}\| &\leq \|\textbf{u}\|+\|\textbf{v}\|
        \end{align*}
\end{enumerate} 

\end{bev}

\section{Frobenius norm}
Definitionen og sætningerne i dette afsnit er baseret på \cite{"Frobenius"}.

I foregående afsnit blev normen af en vektor defineret. Ligeledes kan man definere en norm for matricer, som kaldes \textit{Frobenius normen}. Denne norm er defineret på følgende måde. 

\begin{defn} \textbf{Frobenius normen}\vspace{4pt}\\
    Lad $A = \begin{bmatrix}
        a_{ij}
    \end{bmatrix} \in \R^{n\times n}$. Frobenius normen af $A$ er defineret ved
    %
    \begin{align}
      \norm{A}_F = \sqrt{ \sum_{i=1}^n \sum_{j=1}^n \abs{a_{ij}}^2 }
    \end{align}
\end{defn}

\autoref{sæt:frobenius_egenskab_1} og \autoref{sæt:frobenius_egenskab_2} er egenskaber tilhørende normer, som vil blive benyttet fremadrettet i dette projekt. 

\begin{thmx}\textbf{} \label{sæt:frobenius_egenskab_1}\\
Lad $A\in \R^{n\times n}$ og $\textbf{x}\in \R^n$. Da gælder det, at
    \begin{align*}
        \norm{A\textbf{x}}\leq \norm{A}_F \norm{\textbf{x}}
    \end{align*}
\end{thmx}
a
\begin{bev}\textbf{}\\
I dette bevis betegnes søjlerne i $A$ som $\textbf{a}_k$ for $k=1, ..., n$.
    \begin{align*}
        A &= \begin{bmatrix} \textbf{a}_1 & \textbf{a}_2 & \cdots & \textbf{a}_n \end{bmatrix}\\
        \norm{A\textbf{x}}&=\norm{\textbf{a}_1x_1+\textbf{a}_2x_2+\cdots + \textbf{a}_nx_n}\\
    \intertext{Trekantsuligheden fra \autoref{sæt:norm2} benyttes.}
        \norm{\textbf{a}_1x_1+\textbf{a}_2x_2+\cdots + \textbf{a}_nx_n}&\leq \norm{\textbf{a}_1x_1}+\norm{\textbf{a}_2x_2}+\cdots + \norm{\textbf{a}_nx_n}
    \intertext{\autoref{sæt:norm2} bruges til at faktorisere $x_k$ ud af normerne.}
        \norm{\textbf{a}_1x_1}+\norm{\textbf{a}_2x_2}+\cdots + \norm{\textbf{a}_nx_n}&=\norm{\textbf{a}_1}\abs{x_1}+\norm{\textbf{a}_2}\abs{x_2}+\cdots +\norm{\textbf{a}_n}\abs{x_n}\\
        &=\sum_{k=1}^n\norm{\textbf{a}_k}\abs{x_k}\\
    \intertext{Dette kan omskrives til det indre produkt. Det bemærkes, at $\abs{\textbf{x}}$ er ensbetydende med at tage absolutværdien af hver indgang i \textbf{x}.}
        \sum_{k=1}^n\norm{\textbf{a}_k}\abs{x_k}&= \abs{\left\langle  \left[ \ \norm{\textbf{a}_1}, \norm{\textbf{a}_2}, \cdots, \norm{\textbf{a}_n}\right]^T, \ \abs{\textbf{x}} \right\rangle}\\
    \intertext{Cauchy-Schwarz' ulighed (se \autoref{sæt:cauchy-schwarz}) benyttes.}
        \abs{\left\langle  \left[ \ \norm{\textbf{a}_1}, \norm{\textbf{a}_2}, \cdots, \norm{\textbf{a}_n}\right]^T, \ \abs{\textbf{x}} \right\rangle}&\leq \sqrt{\sum_{k=1}^n\norm{\textbf{a}_k}^2}\sqrt{\sum_{k=1}^n\abs{x_k}^2}\\
        &= \sqrt{\sum_{k=1}^n\sum_{i=1}^n{\abs{a_{ik}}}^2}\sqrt{\sum_{k=1}^n\abs{x_k}^2} \\
        &=\norm{A}_F\norm{\bm{x}}
    \intertext{Altså gælder det, at}
        \norm{A\textbf{x}} &\leq \norm{A}_F\norm{\bm{x}}
    \end{align*}
\end{bev}

\begin{thmx} \textbf{} %Ny sætning
\label{sæt:frobenius_egenskab_2}
\newline
Lad $A,B\in \R^{n\times n}$. Da gælder det, at
$$\norm{AB}_F \leq \norm{A}_F \norm{B}_F$$
\end{thmx}

\begin{bev} \textbf{} %Nyt bevis
\newline 
For at bevise dette undersøges Frobenius normen af $AB$ kvadreret. Når to matricer multipliceres med hinanden, multiplicerer man hver søjle i den første matrice med hver række i den anden matrice. Derfor gælder nedenstående lighed.
\begin{align*}
    \norm{AB}_F^2 &=\sum_{i=1}^n \sum_{j=1}^n\abs{\langle \textbf{a}_i,\textbf{b}_j \rangle}^2\\
    \intertext{Cauchy-Schwarz' ulighed (se \autoref{sæt:cauchy-schwarz}) benyttes.}
    \sum_{i=1}^n \sum_{j=1}^n\abs{\langle \textbf{a}_i,\textbf{b}_j \rangle}^2&\leq \sum_{i=1}^n \sum_{j=1}^n \norm{\textbf{a}_i}^2\norm{\textbf{b}_j}^2\\
    \intertext{Da $\textbf{a}_i$ ikke afhænger af $j$, og $\textbf{b}_j$ ikke afhænger af $i$, kan summerne opdeles.}
    \sum_{i=1}^n \sum_{j=1}^n \norm{\textbf{a}_i}^2\norm{\textbf{b}_j}^2&=\sum_{i=1}^n\norm{\textbf{a}_i}^2\sum_{j=1}^n\norm{\textbf{b}_j}^2\\
    \intertext{Normen udskrives for $\mathbf{a}_i$ og $\mathbf{b}_j$.}
    \norm{AB}_F^2 &\leq \sum_{i=1}^n\norm{\textbf{a}_i}^2\sum_{j=1}^n\norm{\textbf{b}_j}^2\\ 
    &=\sum_{i=1}^n \sum_{k=1}^n\abs{a_{ik}}^2\sum_{j=1}^n \sum_{k=1}^n\abs{b_{jk}}^2\\
    &= \norm{A}_F^2 \norm{B}_F^2
    \intertext{Ved at tage kvadratroden på begge sider gælder følgende}
    \sqrt{\norm{AB}_F^2} &\leq \sqrt{\norm{A}_F^2 \norm{B}_F^2}\\
    &\Updownarrow\\
    \norm{AB}_F &\leq \norm{A}_F \norm{B}_F
\end{align*}
\end{bev}




