\section{Ligevægtspunkt}
Dette afsnit er baseret på \cite[]{"Arne"}.

Lad et autonomt lineært system af første ordensdifferentialigninger være givet ud fra \autoref{def:definition_lineært_autonomt_system}.
%
\begin{align*}
    \frac{d\textbf{x}}{dt} = \textbf{f}(\textbf{x})
\end{align*}
%
Da er begyndelsesværdiproblemet for det autonome system givet ved,
\begin{align*}
    \frac{d\textbf{x}}{dt} = \textbf{f}(\textbf{x}) \quad \textbf{x}(t_0) = \textbf{x}_0 
\end{align*}
%
Lad $\textbf{f}(\textbf{x}_0) = \textbf{0}$. Da vil $\textbf{x}(t)= \textbf{x}_0$, for alle $t\in I$, være løsning til begyndelsesværdiproblemet, og $\textbf{x}_0$ kaldes et \textit{ligevægtspunkt}. Generelt defineres ligevægtspunktet som følgende. 

%Et ligevægtspunkt til dette begyndelsesværdiproblem er en vektor, som opfylder nedenstående definition.


%Lad $\textbf{x}(t_0)=\textbf{x}_0$ være begyndelsesbetingelsen for det autonome system. Hvis $f(\textbf{x}_0)=0$ så gælder, at den konstante vektorfunktion $\textbf{x}(t)=\textbf{x}_0$ for alle $t$ er løsning til begyndelsesværdiproblemet


%Løsningen til et ligevægtspunkt ses ud fra den konstante vektorfunktion $\textbf{x}(t) = \textbf{x}^*$, som også er en løsning til begyndelsesværdiproblemet $\textbf{x}(0) = \textbf{x}^*$.



\begin{defn} \textbf{Ligevægtspunkt for et autonomt system}\label{ligevægtspunkt_defn}\\
    %$\textbf{x}^*$ er et ligevægtspunkt, såfremt det gælder, at
    $\textbf{x}_0$ kaldes et ligevægtspunkt og noteres $\textbf{x}^*$, såfremt det gælder, at
    \begin{align*}
        \textbf{f}(\textbf{x}_0) = \textbf{0}
    \end{align*}
\end{defn} 
