\section{Stabilitet ved egenværdier}
Dette afsnit er baseret på \cite[s. 30-35]{"Morris_and_Stephen"}
\begin{thmx}\textbf{}\label{sæt:Løsning_for_komplekse_egenværdier}
\newline
Lad $A\in \R^{2\times 2}$ og $\lambda\in \C$ være en egenværdi til A med tilhørende egenvektor $\textbf{v}\in \C^2$. Så gælder det, at
\begin{align}\label{eq:tilhørende_egenvektor}
    x(t) = e^{t\lambda}\textbf{v}
\end{align}
er en løsning til 
\begin{align}\label{eq:diff_ligning_stabilitet_ved_egenværdi}
    \frac{d\textbf{x}}{dt} = A\textbf{x}
\end{align}
hvor $t\in \I$ og $\textbf{x}\in C^1(I, \R^2)$.
\end{thmx}

\begin{bev} \textbf{}
\newline
Dette vises ved at differentiere $\textbf{x}(t) = e^{t\lambda}\textbf{v}$.

\begin{align*}
    \frac{dx}{dt} = \lambda e^{t\lambda} \textbf{v} = e^{t\lambda} (\lambda\textbf{v}) =  e^{t\lambda} (A\textbf{v}) = Ae^{t\lambda} \textbf{v} = A\textbf{x}
\end{align*}

Det er hermed bevist, at $\textbf{x}(t) = e^{t\lambda}\textbf{v}$ er en løsning til \eqref{eq:diff_ligning_stabilitet_ved_egenværdi}.
\end{bev}


\begin{thmx}\textbf{} \label{sæt_x(t)_skrevet_lambda_egenvektorer}
\newline
Lad $A\in \R^{2\times 2}$ og $\lambda_1\neq \lambda_2 \in \R$ være egenværdier til $A$ med de tilhørende egenvektorer $\textbf{v}_1, \textbf{v}_2\in \R^2$. Så gælder det, at 
\begin{align}\label{eq_x(t)_skrevet_lambda_egenvektorer}
    \textbf{x}(t) = c_1e^{t\lambda_1}\textbf{v}_1 + c_2e^{t\lambda_2}\textbf{v}_2
\end{align}
er en løsning til 
\begin{align}\label{eq:begyndelsesværdiproblem_stabilitet_ved_egenværdi}
    \frac{d\textbf{x}}{dt} = A\textbf{x}, \quad \mathbf{x}(t_0) = \mathbf{x}_0
\end{align}

hvor $t,c_1,c_2 \in \R$ og $\textbf{x}\in C^1(\R,\R^2)$.

\end{thmx}

\begin{bev} \textbf{}
\newline
Ud fra eksistens- og entydighedssætning (\autoref{sæt:eksistens_og_entydighed} vides det, at der er en entydig løsning til \eqref{eq:begyndelsesværdiproblem_stabilitet_ved_egenværdi}. Derfor er det nok at vise eksistens af løsningen. 
$\textbf{x}(t) = c_1e^{t\lambda_1}\textbf{v}_1 + c_2e^{t\lambda_2}\textbf{v}_2$ differentieres.

\begin{align*}
    \frac{d\textbf{x}}{dt} &= \left(c_1 e^{t\lambda_1}\textbf{v}_1\right)' + \left(c_2e^{t\lambda_2}\textbf{v}_2\right)'\\
    &= c_1 \left( e^{t\lambda_1}\textbf{v}_1\right)' + c_2\left(e^{t\lambda_2}\textbf{v}_2\right)'\\
    &= c_1 \lambda_1 e^{t\lambda_1}\textbf{v}_1 + c_2 \lambda_2 e^{t\lambda_2}\textbf{v}_2\\
    &= c_1 e^{t\lambda_1}\lambda_1\textbf{v}_1 + c_2 e^{t\lambda_2} \lambda_2\textbf{v}_2\\
    &= c_1 e^{t\lambda_1}A\textbf{v}_1 + c_2 e^{t\lambda_2} A\textbf{v}_2\\
    &= A \left(c_1 e^{t\lambda_1}\textbf{v}_1 + c_2 e^{t\lambda_2} \textbf{v}_2\right)\\
    &=A\textbf{x}
\end{align*}

Begyndelsesbetingelsen verificeres. $x(0)=x_0$ kan opstilles som en linearkombination af de to egenvektorer, da $\lambda_1\neq\lambda_2$ medfører, at $\textbf{v}_1$ og $\textbf{v}_2$ er lineært uafhængige, og dermed er basis for $\R^2$. Dette kan altså opstilles på følgende måde.
\begin{align*}
    x_0 = \alpha \textbf{v}_1 + \beta \textbf{v}_2
\end{align*}

Begyndelsesbetingelsen indsættes i \eqref{eq_x(t)_skrevet_lambda_egenvektorer}.

\begin{align*}
    x(0) &= c_1e^{0\cdot \lambda_1}\textbf{v}_1 + c_2e^{0\cdot \lambda_2}\textbf{v}_2 \\
    &= c_1 \textbf{v}_1 + c_2 \textbf{v}_2
\intertext{$c_1 = \alpha$ og $c_2 = \beta$ indsættes.}
    x(0) &= \alpha \textbf{v}_1 + \beta \textbf{v}_2 = x_0
\end{align*}
\autoref{sæt_x(t)_skrevet_lambda_egenvektorer} er hermed bevist.

\end{bev}


