\section{Koordinatskifte}

For at kunne analysere stabiliteten af et system ud fra egenværdierne af koefficientmatricen, er det en fordel at lave et \textit{koordinatskift}. Dette gøres for at få koefficientmatricen for systemet på \textit{kanonisk form}, hvorved stabiliteten kan beskrives ud fra egenværdierne.
Kanonisk form defineres som følgende for $2\times 2$ matricer
\begin{defn}\textbf{Kanonisk form}\\
    En $2\times 2$ matrice siges at være på kanonisk form, såfremt den antager en af følgende former
%    
    \begin{align*}
        \begin{bmatrix}
            \lambda_1   &  0\\
            0           &  \lambda_2
        \end{bmatrix}, \quad
        \begin{bmatrix}
            \phantom{-}\alpha & \beta \\
            -\beta & \alpha
        \end{bmatrix}, \quad
        \begin{bmatrix}
            \lambda & 1\\
            0       & \lambda
        \end{bmatrix}
    \end{align*}
\end{defn}

\begin{thmx}\textbf{Koordinatskift}\\
Lad $T\in\R^{2\times 2}$ være invertibel og $A\in \R^{2\times 2}$.
%
\begin{align}\label{eq:y(t)_og_invertibel_T_koordinatskifte}
    \frac{d\textbf{y}}{dt} = T^{-1}AT \textbf{y}
\end{align}
%
Hvis $\textbf{y}(t)\in C^1(\R, \R^2)$ er en løsning til \eqref{eq:y(t)_og_invertibel_T_koordinatskifte}, så gælder det, at 
\begin{align}\label{eq:x(t)_koordinatskifte}
    \textbf{x}(t)=T\textbf{y}(t)
\end{align}
%
er løsning til %$\displaystyle \frac{d\textbf{x}}{dt}= A\textbf{x}$
%
\begin{align}\label{eq:x'(t)=Ax_koordinatskifte}
    \displaystyle \frac{d\textbf{x}}{dt}= A\textbf{x}
\end{align}

%hvis y(t) er løsning så er
%x(t)=Ty(t) løsning til x'(t) = Ax(t)

\end{thmx}

\begin{bev}\textbf{}\\
$\textbf{x}(t)$ differentieres.
    \begin{align*}
        \frac{dx}{dt}   &= \left(Ty(t)\right)' 
                         = Ty'(t)
                         = T\left(T^{-1}AT \textbf{y}\right)
                         = A(T\textbf{y})
                         = A\textbf{x}
    \end{align*}
Det er hermed bevist, at \eqref{eq:x(t)_koordinatskifte} er en løsning til \eqref{eq:x'(t)=Ax_koordinatskifte}.  
\end{bev}

Ud fra dette ses det, at matricen $T$ konverterer løsninger for  \eqref{eq:y(t)_og_invertibel_T_koordinatskifte} til løsninger for \eqref{eq:x'(t)=Ax_koordinatskifte}. Omvendt konverterer den inverse matrice $T^{-1}$, løsninger for \eqref{eq:x'(t)=Ax_koordinatskifte}  til løsninger for \eqref{eq:y(t)_og_invertibel_T_koordinatskifte}. 

%linear system into one whose coefficient matrix is different. What we hope to be able to do is find a linear map T that converts the given system into a system of the form Y  = (T −1AT)Y that is easily solved. And, as you may have guessed, we can always do this by finding a linear map that converts a given linear system to one in canonical form.

\begin{thmx}\textbf{Koordinatskift til kanonisk form}\\
    Lad $T\in\R^{2\times 2}$ være invertibel og lad $A\in \R^{2\times 2}$ være en matrice med egenværdier $\lambda_1$ og $\lambda_2$ med tilhørende egenvektorer $\textbf{v}_1, \textbf{v}_2\in \R^2$. Så gælder det, at $T^{-1}AT$ er på følgende kanoniske form,
    
    
    \begin{enumerate}
        
        \item hvis $\lambda_1,\lambda_2 \in \R$ og $\lambda_1 \neq \lambda_2$ er 
        \begin{align*}
            T^{-1}AT = \begin{bmatrix}
                            \lambda_1   &  0\\
                            0           &  \lambda_2
                        \end{bmatrix}
        \end{align*}
        
        \item hvis $\lambda_1, \lambda_2 \in \C$, hvor $\lambda = \alpha \pm i \beta$, er
        \begin{align*}
            T^{-1}AT = \begin{bmatrix}
                            \phantom{-}\alpha &  \beta\\
                            - \beta           &  \alpha
                        \end{bmatrix}
        \end{align*}
        
        \item hvis $\lambda_1, \lambda_2 \in \R$ og $\lambda_1 = \lambda_2$ er 
        \begin{align*}
            T^{-1}AT = \begin{bmatrix}
                            \lambda\phantom{_1}   &  k\phantom{_1}\\
                            0\phantom{_2}           &  \lambda\phantom{_2}
                        \end{bmatrix}
        \end{align*}
        hvor $k=0$, hvis $A$ har to lineært uafhængige egenvektorer tilhørende $\lambda$, og $k=1$, hvis $A$ kun har en egenvektor tilhørende $\lambda$.
    \end{enumerate}
\end{thmx}

\begin{bev}\textbf{}\\
    \begin{itemize}
        \item \textbf{Bevis for punkt 1}\\
        Lad $\lambda_1, \lambda_2 \in \R$ være egenværdier for $A\in \R^{2\times 2}$ med de tilhørende egenvektorer $\textbf{v}_1, \textbf{v}_2 \in \R^2$, hvor $\lambda_1\neq \lambda_2$, hvilket medfører, at $\textbf{v}_1$ og $\textbf{v}_2$ er lineært uafhængige. Lad $T$ være den invertible matrice med søjlerne $\textbf{v}_1$ og $\textbf{v}_2$. Så må der gælde, at $T\textbf{e}_j=\textbf{v}_j$ for $j=1,2$, hvor  $\textbf{e}_j$ betegner den j'te standardvektor i $\R^2$. Der må derfor også gælde, at $T^{-1}\textbf{v}_j=\textbf{e}_j$.
        
        Ud fra dette kan søjlerne i $T^{-1}AT$ konstrueres.
        \begin{align*}
            \left(T^{-1}AT\right)\textbf{e}_j 
            = T^{-1}A\textbf{v}_j 
            = T^{-1}\lambda_j\textbf{v}_j
            = \lambda_j T^{-1} \textbf{v}_j
            = \lambda_j \textbf{e}_j, \quad j=1,2
        \end{align*}
        
        Matricen $T^{-1}AT$ kan herudfra opstilles.
        %
        \begin{align*}
            T^{-1}AT = 
            \begin{bmatrix}
                \lambda_1 & 0\\
                0 & \lambda_2
            \end{bmatrix}
        \end{align*}
        Det er hermed bevist, at $T^{-1}AT$ er på kanonisk form, når $A$ har to forskellige reelle egenværdier.
    \end{itemize}
    
    \begin{itemize}
        \item \textbf{Bevis for punkt 2}\\
        Lad $A\in \R^{2\times 2}$ have et par af komplekse konjugerede egenværdier, $\lambda = \alpha \pm i\beta$, hvor $\beta \neq 0$. En kompleks egenvektor tilhørende $\lambda = \alpha + i \beta $ kan opskrives som $\textbf{v}_1 + i \textbf{v}_2$, hvor $\textbf{v}_1$ og $\textbf{v}_2$ er lineært uafhængige reelle vektorer. Hvis $\textbf{v}_1$ og $\textbf{v}_2$ ikke er lineært uafhængige, vil det gælde, at $\textbf{v}_1=c\textbf{v}_2$ for $c\in \R$, hvilket vil føre til en modstrid. Denne modstrid fås, ved at opskrive $ A\left(\textbf{v}_1 + i\textbf{v}_2\right)$ på følgende to forskellige måder
        \begin{align*}
            A\left(\textbf{v}_1 + i\textbf{v}_2\right) 
            &= \left(\alpha + i\beta\right)\left(\textbf{v}_1 + i \textbf{v}_2\right) 
            = \left(\alpha + i\beta\right) \left(c\textbf{v}_2+i\textbf{v}_2\right) 
            = \left(\alpha + i\beta\right)\left(c + i\right)\textbf{v}_2
    \intertext{og}
            A\left(\textbf{v}_1 + i\textbf{v}_2\right) 
            &= A\left(c\textbf{v}_2 + i\textbf{v}_2\right)
            = \left(c+i\right)A\textbf{v}_2
        \end{align*}
        De to udtryk er lig hinanden, hvilket betyder
        \begin{align*}
            \left(\alpha + i\beta\right)\left(c + i\right)\textbf{v}_2 &= \left(c+i\right)A\textbf{v}_2\\
            &\Updownarrow\\
            \left(\alpha + i\beta\right)\textbf{v}_2 &= A\textbf{v}_2
        \end{align*}
        Herudfra er der en modstrid, da venstresiden er kompleks, hvorimod højresiden er reel. Derfor er $\textbf{v}_1$ og $\textbf{v}_2$ lineært uafhængige. Ud fra $A\textbf{v} = \lambda \textbf{v}$ må det for den komplekse egenvektor $\textbf{v}_1 + i\textbf{v}_2$ tilhørende $\lambda = \alpha + i\beta$ gælde, at
    %
        \begin{align*}
            A(\textbf{v}_1 + i\textbf{v}_2) &= (\alpha + i\beta)(\textbf{v}_1 + i\textbf{v}_2)\\
            &\Updownarrow\\
            A\textbf{v}_1 + iA\textbf{v}_2 &= \alpha\textbf{v}_1 - \beta \textbf{v}_2 + i(\beta\textbf{v}_1 + \alpha \textbf{v}_2)
        \end{align*}
    %
        Udtrykket opdeles i den reelle og imaginære del og skrives som følgende ligningssystem
    %
        \begin{align*}
            A\textbf{v}_1 &= \alpha \textbf{v}_1 - \beta \textbf{v}_2   \\
            A\textbf{v}_2 &= \beta \textbf{v}_1 + \alpha \textbf{v}_2
        \end{align*}  
    %
        Lad $T$ være den invertible matrice med søjlerne $\textbf{v}_1$ og $\textbf{v}_2$. Så må der gælde, at $T\textbf{e}_j = \textbf{v}_j $ for $j=1,2$.
        
        Ud fra dette kan søjlerne i $T^{-1}AT$ konstrueres. Den første søjle konstrueres som følgende
    %
        \begin{align*}
            \left(T^{-1}AT\right)\textbf{e}_1 
            &= T^{-1}A\textbf{v}_1
            = T^{-1} \left(\alpha \textbf{v}_1 - \beta \textbf{v}_2\right)
            = \alpha \textbf{e}_1 - \beta \textbf{e}_2
            \intertext{og ligeledes for den anden søjle}
            \left(T^{-1}AT\right)\textbf{e}_2 
            &= T^{-1}A\textbf{v}_2
            = T^{-1} \left(\beta \textbf{v}_1 + \alpha \textbf{v}_2\right)
            = \beta \textbf{e}_1 + \alpha \textbf{e}_2
        \end{align*}
    %
        Matricen $T^{-1}AT$ kan herudfra opstilles. 
    %
        \begin{align*}
            T^{-1}AT =
            \begin{bmatrix}
               \phantom{-} \alpha & \beta\\
                - \beta           & \alpha
            \end{bmatrix}
        \end{align*}
    %
        Det er hermed bevist, at $T^{-1}AT$ er på kanonisk form, når $A$ har et par af komplekse konjugerede egenværdier.
    \end{itemize}

    \begin{itemize}
        \item \textbf{Bevis for punkt 3}\\
        Lad $A\in \R^{2\times 2}$ have en enkelt egenværdi $\lambda = \lambda_1= \lambda_2$, hvor $\lambda \in \R$. Hvis der er et par reelle egenvektorer, tilhørende $\lambda$, $\textbf{v}_1$ og $\textbf{v}_2$, som er lineært uafhængige, vil $T^{-1}AT$ kunne konstrueres ligesom i punkt 1, altså
    %
        \begin{align}\label{eq:bevis_koordinatskifte_k=0}
            T^{-1}AT = 
            \begin{bmatrix}
                \lambda & 0\\
                0       & \lambda
            \end{bmatrix}
        \end{align}
    %
        som er på kanonisk form.\\
        Hvis der derimod kun er en egenvektor tilhørende $\lambda$ således, at $\textbf{v} = \textbf{v}_1 = c\textbf{v}_2$ for $c\in \R$, vil den kanoniske form være anderledes. 
        Eftersom $\lambda$ er en egenværdi og $v$ den tilhørende egenvektor må følgende gælde
    %
        \begin{align*}
            A\textbf{v}               &= \lambda \textbf{v}\\
                                      &\Updownarrow\\
            (A-\lambda I_2)\textbf{v} &= \textbf{0}
        \end{align*}
    %
        Lad $\textbf{w}\in \R^2$ være en vektor, som er lineært uafhængig af $\textbf{v}$.
        Det vides fra Cayley-Hamiltons sætning (\autoref{lem:Cayley_Hamilton_Sætning}), at $(A-\lambda I_2)^2=O$, og derfor kan $\textbf{v}$ defineres som
     %   
        \begin{align}\label{eq:bevis_koordinatskifte_bevis_3_v=(A-lam_I)w}
            \textbf{v} = (A-\lambda I_2)\textbf{w}
        \end{align}
    %
        $A\textbf{w}$ isoleres i \eqref{eq:bevis_koordinatskifte_bevis_3_v=(A-lam_I)w}
    %
        \begin{align*}
            \textbf{v}  = (A-\lambda I_2)\textbf{w}
                        \Leftrightarrow
            \textbf{v}  = A\textbf{w}-\lambda\textbf{w}
                        \Leftrightarrow
            A\textbf{w} = \textbf{v} + \lambda \textbf{w}
        \end{align*}
    %   
        Lad $T$ være den invertible matrice med søjlerne $\textbf{v}$ og $\textbf{w}$. Så må der gælde, at $T\textbf{e}_j= v_j$. Den første søjle i $T^{-1}AT$ konstrueres som følgende
        \begin{align*}
            (T^{-1}AT)\textbf{e}_1 
            = T^{-1}A\textbf{v} 
            = T^{-1}\lambda \textbf{v} 
            = \lambda \textbf{e}_1 
        \intertext{og ligeledes den anden søjle}
            (T^{-1}AT)\textbf{e}_2 
            = T^{-1}A\textbf{w} 
            = T^{-1}(\textbf{v} + \lambda \textbf{w}) 
            = \textbf{e}_1 + \lambda \textbf{e}_2
        \end{align*}
    %   
        Matricen $T^{-1}AT$ kan herudfra opstilles.
        \begin{align}\label{eq:bevis_koordinatskifte_k=1}
            T^{-1}AT = \begin{bmatrix}
                            \lambda   &  1\\
                            0         &  \lambda
                       \end{bmatrix}
        \end{align}
    %
        De to tilfælde \eqref{eq:bevis_koordinatskifte_k=0} og \eqref{eq:bevis_koordinatskifte_k=1} kan generaliseres således at
    %
        \begin{align*}
            T^{-1}AT = \begin{bmatrix}
                            \lambda   &  k\\
                            0         &  \lambda
                       \end{bmatrix}
        \end{align*}
    %
        hvor $k=0$, hvis $A$ har to lineært uafhængige egenvektorer tilhørende $\lambda$, og $k=1$, hvis $A$ kun har en egenvektor tilhørende $\lambda$.\\
        Det er hermed bevist, at $T^{-1}AT$ er på kanonisk form, når $A$ har en reel egenværdi.
    \end{itemize}
\end{bev}



\begin{eks}
    Lad $A\in R^{2\times 2}$ være en matrice givet ved
%    
    \begin{align*}
        A = \begin{bmatrix}
                -1 & \phantom{-}0\\
                \phantom{-}1 & -2
        \end{bmatrix}
    \end{align*}
%
    Ud fra matricens karakteristiske polynomium, fås $\lambda_1 = -1$ og $\lambda_2 = -2$. 
    Ud fra egenværdierne findes egenvektorerne, hvor 
    $\textbf{v}_1 = \begin{bmatrix}
        1\\1
    \end{bmatrix}$ og
    $\textbf{v}_2 = \begin{bmatrix}
        0\\1
    \end{bmatrix}$ er egenvektorer til henholdsvis $\lambda_1$ og $\lambda_2$,
%
    Hvis $A$ er koefficientmatrice til et system af differentialligninger, kan systemet omskrives til et system, hvor koefficientmatricen antager kanonisk form. $T$ vælges til at være matricen, med $\textbf{v}_1$ og $\textbf{v}_2$ som søjler således at
%    
    \begin{align*}
        T = \begin{bmatrix}
                1 & 0\\
                1 & 1
            \end{bmatrix}
    \end{align*}    
%
    Da $T$ per definition er invertibel, kan $T^{-1}$ udregnes til følgende
%    
    \begin{align*}
      T^{-1} = \begin{bmatrix}
                    \phantom{-}1 & 0\\
                    -1 & 1
                \end{bmatrix}
    \end{align*}    
%   
    Nu kan den den kanoniske form $T^{-1}AT$ udregnes
%
    \begin{align*}
        T^{-1}AT =
            \begin{bmatrix}
                \phantom{-}1 & 0\\
                    -1 & 1
            \end{bmatrix}
            \begin{bmatrix}
                -1 & \phantom{-}0\\
                \phantom{-}1 & -2
            \end{bmatrix}
            \begin{bmatrix}
                1 & 0\\
                1 & 1
            \end{bmatrix} 
            =
            \begin{bmatrix}
                -1 & \phantom{-}0\\
                \phantom{-}0 & -2
            \end{bmatrix}
    \end{align*}
    \textbf{OBS} Læs kommentar i koden
    %
    % Er dette eksempel ikke lidt redundant? For vi ender jo bare med en matrice af den form der står i sætning 6,4,3. Dvs. der aldrig vil være en grund til at udregne T og T^-1 og egenvektorerne da vi har bevist matematikken og har vist at det kommer til at være på kanonisk form. 
    %
    %Så et eksempel ville give mere mening hvis man lavede det for et specifikt system, og derved konverterede løsningen x(t) til Ty(t) med koordinatskifte. Tror at det ville give mening at enten - lave eksemplet om, eller - lave endnu et eksempel når vi kommer til fasediagrammer. 
    %
\end{eks}