\section{Fasediagrammer}

Dette afsnit er baseret på \cite[s. 145]{"Ronald_Shone"}.

%For et autonomt system, $\displaystyle \frac{d\textbf{x}}{dt} =  A\textbf{x}$, vil der for ethvert $t\in I$ eksistere et $\textbf{x}(t)$, hvor løsningskurven er entydigt bestemt for begyndelsesbetingelsen $\textbf{x}(t_0)=\textbf{x}_0$.

For begyndelsesværdiproblemet til det autonome system
\begin{align*}
    \frac{d\textbf{x}}{dt}= \textbf{f}(\textbf{x}), \quad \textbf{x}(t_0)=\textbf{x}_0 
\end{align*}
%
hvor $\textbf{x}\in \R^2$, vil der for ethvert $t \in I$ eksistere et punkt $\left(x_1(t),x_2(t)\right)$ i $(x_1,x_2)$-planen. Løsningskurven består af mængden af alle disse punkter, altså mængden $\left\{\left(x_1(t),x_2(t)\right)\hspace{-1.4pt}\big|t\in I\right\}$.                                                   % 300 iq ^

%For et system af to variable, $x_1$ og $x_2$, vil der for ethvert $t \in I$ eksistere et punkt $(x_1,x_2)$ i $(x_1,x_2)$-planen, hvor løsningskurven er entydigt bestemt for begyndelsesbetingelsen $\left(x_1(t_0),x_2(t_0)\right)=(x_{2_0},x_{2_0})$. Altså eksisterer der en entydig bestemt funktion $x_2=\Phi(x_1)$, som opfylder begyndelsesbetingelsen $\Phi(x_{1_0})=x_{2_0}$. 

Fasediagrammer kan anvendes til at visualisere løsningskurven i $(x_1,x_2)$-planen.
Løsningskurven kaldes for systemets \textit{bane}, og $(x_1,x_2)$-planen, hvor banen er defineret, kaldes for \textit{faseplanen}. Mængden af alle løsningskurver kaldes for \textit{faseportrættet}. \textit{Retningsfeltet} for systemet er udgjort af tangentvektorer til løsningskurver i et passende antal punkter i $(x_1,x_2)$-planen.

I nedenstående eksempel visualiseres de ovenstående begreber.
\begin{eks}\textbf{}
\newline
Lad følgende begyndelsesværdiproblem for det autonome system være givet.
\begin{align*}
    \frac{d\textbf{x}}{dt}=\begin{bmatrix} 1 & -4 \\ 2 & -5 \end{bmatrix}\textbf{x}, \quad \textbf{x}(0)=\begin{bmatrix} 1 \\ 0 \end{bmatrix}
\end{align*}
% \begin{align*}
%     \frac{dx_1}{dt}&=x_1-4x_2, \phantom{2}\quad  x_{1}(0)=1 \\
%     \frac{dx_2}{dt}&=2x_1-5x_2, \quad x_{2}(0)=0
% \end{align*}
Da dette er et autonomt lineært system af første ordens differentialligninger, kan løsningen bestemmes jævnfør \autoref{sæt:løsning_til_homogen_system_af_første_ordens_differentialligninger} og \autoref{sæt:løsning_2_x_2_ved_egenværdier_etA_beregning}.
%
\begin{align*}
    \textbf{x}(t)  = \begin{bmatrix} 2e^{-t}-e^{-3t} \\ e^{-t}-e^{-3t} \end{bmatrix}
\end{align*}
% \begin{align*}
%     x_1 &= e^{-t}-e^{-3t} \\
%     x_2 &= 2e^{-t}-e^{-3t}
% \end{align*}
Løsningenskurven illustreres i $(x_1,x_2)$-planen.
%
\begin{figure}[H]
    \centering
    \includegraphics[scale=0.5]{Billeder/faseportræt_eksempel.png}
    \caption{Fasediagram for systemet}
    \label{fig:faseportræt_eksempel}
\end{figure}
%
I \autoref{fig:faseportræt_eksempel} udgør de røde pile retningsfeltet, og den blå kurve udgør systemets bane. Eftersom figuren kun visualiserer løsningen for én begyndelsesbetingelse, vil løsningskurven også udgøre faseportrættet. Ud fra retningsfeltet, kan man redegøre for, hvordan andre løsninger i faseplanen vil opføre sig.
%
\end{eks}