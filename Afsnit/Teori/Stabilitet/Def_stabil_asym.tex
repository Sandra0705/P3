\section{Stabilitets definitioner}
Dette afsnit er baseret på \cite[s. 492-494]{"Edwards_and_Penney"}.

Når ligevægtspunktet er bestemt er det muligt at analysere \textit{stabiliteten} for et system. Stabiliteten af et system beskriver, hvordan systemet opfører sig i forhold til ligevægtspunktet, samt systemets opførsel, når $t \rightarrow \infty$. 

\begin{minipage}\textwidth
\begin{defn}\textbf{Stabilitet} %Ny definition
\newline
For et autonomt system $\displaystyle\frac{d\textbf{x}}{dt}=\textbf{f}(\textbf{x})$ gælder det, at ligevægtspunktet $\textbf{x}^*$ er stabilt, hvis der for ethvert $\textbf{x}_0\in \R^n$ gælder, at
\begin{align*}
\norm{\textbf{x}_0 - \textbf{x}^*} < \delta \Rightarrow \norm{\textbf{x}(t)- \textbf{x}^*} < \varepsilon
\end{align*}
hvor $\delta, \varepsilon>0$. Ligevægtspunktet kaldes ustabilt, hvis det ikke er stabilt.
\end{defn}
\end{minipage}

Det vil sige, at ligevægtspunktet $\textbf{x}^*$ er stabilt, hvis begyndelsesbetingelsen $\textbf{x}_0$ er indenfor en $\delta$-afstand af ligevægtspunktet, $\textbf{x}^*$, og $\textbf{x}(t)$ forbliver indenfor en $\varepsilon$-afstand af ligevægtpunktet $\textbf{x}^*$ for alle $t\in I$. Hvis ligevægtspunktet $\textbf{x}^*$ er ustabilt, gælder der, at $\textbf{x}(t)$ ikke forbliver indenfor en $\varepsilon$-afstand af ligevægtpunktet.

\begin{minipage}\textwidth
\begin{defn}\textbf{Asymptotisk stabilitet} %Ny definition
\newline
For et autonomt system $\displaystyle\frac{d\textbf{x}}{dt}=\textbf{f}(\textbf{x})$ gælder det, at ligevægtspunktet $\textbf{x}^*$ er asymptotisk stabilt, hvis der for alle $\textbf{x}_0 \in \R^n$ gælder, at ligevægtspunktet både er stabilt og der gælder, at
\begin{align*}
\norm{\textbf{x}_0 - \textbf{x}^*} < \delta \Rightarrow \lim_{t\rightarrow \infty} \textbf{x}(t) = \textbf{x}^*
\end{align*}
hvor $\delta>0$.
\end{defn}
\end{minipage}

Altså er ligevægtspunktet asymptotisk stabilt, hvis det både er stabilt, og der gælder, at $\textbf{x}(t)$ nærmer sig ligevægtspunktet, når $t \rightarrow \infty$.
