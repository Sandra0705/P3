\section{Ligevægtspunkt}
Dette afsnit er baseret på \cite[]{"Arne"}\\
Lad $D_f \subseteq \R$ og $f \in C^0(D_f,R^n)$ så kan
\begin{align*}
    \frac{dx_1}{dt} &= f_1(x_1,x_2)\\
    \frac{dx_2}{dt} &= f_2(x_1,x_2)
\end{align*}
hvor $t \in I$ og $\textbf{x} \in C^1(I,D_f)$, betragtes som følgende autonomt system
\begin{align}\label{Ligevægtspunkt_autonome_differentialligning}
    \frac{d\textbf{x}}{dt} = \textbf{f}(\textbf{x}) \quad \textbf{x}(t_0) = \textbf{x}_0. 
\end{align}
Ud fra ovenstående informationer af det autonome differentialligningssystem, er det muligt at finde et \textit{ligevægtspunkt}. 
\begin{defn}\textbf{ligevægtspunkt}\label{ligevægtspunkt_defn}\\
En begyndelsesbetingelse som $\textbf{x}(t_0) = x^*$ kaldes et ligevægtspunkt, såfremt følgende betingelse gælder
\begin{equation}
    \textbf{f}(\textbf{x}^*) =\textbf{ 0}
\end{equation}
Løsning til et ligevægtspunkt ses ud fra den konstante vektorfunktion $\textbf{x}(t) = \textbf{x}*$, som også er en løsning til begyndelsesværdiproblemet $\textbf{x}(0) = \textbf{x}^*$.
\end{defn}