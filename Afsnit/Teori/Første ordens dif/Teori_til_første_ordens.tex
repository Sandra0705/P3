\section{Første ordens differentialligninger}

En første ordens differentialligning er en differentialligning, hvor ordenen af den højeste afledede er én. 

\begin{defn} \textbf{Lineær første ordens differentialligning} %Ny sætning
\newline
En første ordens differentialligning med den ukendte funktion $x$ er 
\begin{align}
    \frac{dx}{dt}=f(t,x)  
\end{align}
hvor $f$ er givet. Ligningen er lineær hvis og kun hvis $f$ er lineær,
\begin{align}
    \frac{dx}{dt}+a(t)x=b(t)
\end{align}
Den lineære differentialligning har konstante koefficienter, hvis $a(t)=a$ og $b(t)=b$. Ellers har den lineære differentialligning variable koefficienter.
\end{defn}






