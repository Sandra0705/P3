\section{Første ordens differentialligninger}

I dette afsnit vil differentialligninger af første orden blive introduceret. 
En første ordens differentialligning er defineret ud fra \autoref{def:generel_differentialligning}, hvor $n=1$. 
\begin{align}
    \frac{dx}{dt}=f(t,x)  
\end{align}

I dette projekt vil der blive arbejdet med lineære differentialligninger. En lineær første ordens differentialligning defineres ved følgende. \\
\begin{minipage}\textwidth
\begin{defn} \textbf{Lineær første ordens differentialligning} %Ny sætning
\newline
Lad $a(t), b(t)\in C^0(I,\R)$. Lad $f(t,x)=a(t)x+b(t)$, hvorom det gælder, at $D_f \subseteq \R \times \R$ er en åben ikke-tom delmængde og $f \in C^0(D_f,\R)$. Så kan en lineær første ordens differentialligning med den ukendte funktion $x$ udtrykkes ved
\begin{align}
    \frac{dx}{dt}=a(t)x+b(t)
\end{align}
hvor $t \in I$ og $x \in C^1(\I,\R)$. 
% $a(t)$ og $b(t)$ betegner variable koefficienter. Hvis $a(t)=a$ og $b(t)=b$ for alle $t\in I$, har den lineære differentialligning konstante koefficienter.
$a(t)$ og $b(t)$ kaldes variable koefficienter. Hvis $a(t)=a$ og $b(t)=b$ for alle $t\in I$, kaldes koefficienterne konstante


\end{defn}
\end{minipage}





