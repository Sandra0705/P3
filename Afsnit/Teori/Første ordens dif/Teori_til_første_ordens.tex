\section{Første ordens differentialligninger}

I dette afsnit vil differentialligninger af første orden blive introduceret. 
En første ordens differentialligning er defineret ud fra \autoref{def:generel_differentialligning}, hvor $n=1$. 
\begin{align}
    \frac{dx}{dt}=f(t,x)  
\end{align}

I dette projekt vil der blive arbejdet med lineære differentialligninger. En lineær første ordens differentialligning defineres ved følgende. \\
\begin{minipage}\textwidth
\begin{defn} \textbf{Lineær første ordens differentialligning} %Ny sætning
\newline
Lad $D_f \subseteq \R \times \R$ være en åben ikke-tom delmængde. Lad $f \in C^1(D_f,\R)$. En lineær første ordens differentialligning med den ukendte funktion $x$ er udtrykt ved
\begin{align}
    \frac{dx}{dt}=a(t)x+b(t)
\end{align}
hvor $a(t), b(t) \in C^0(\R)$ og betegner variable koefficienter. Hvis $a(t)=a$ og $b(t)=b$ for alle $t\in I$, har den lineære differentialligning konstante koefficienter.
\end{defn}
\end{minipage}





