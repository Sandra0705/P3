\section{Introduktion til differentialligninger}
Først defineres henholdsvis Leibniz' og Lagranges notation for differentialkvotienter. Disse er de to notationer, som vil blive anvendt i dette projekt.

\begin{defn}\textbf{Leibniz' notation}
\newline
Lad $I = \{i \ | \ a \leq i \leq b\}$ være et interval, hvor $a,b,i\in \R$. Lad $t \in \I $ og $x:I \to \R$. Da vil differentialkvotienten til $x$ være givet ved
%
\begin{align*}
    \frac{dx}{dt}=\lim_{t \to \infty}\frac{\Delta x}{\Delta t}
\end{align*}

\end{defn}
Lagranges notation er en anden måde at udtrykke differentialkvotienter på. Dette betyder, at $\frac{dx}{dt}=x'(t)=x'$, hvilket kan blive generaliseret til $n$'te afledede ved $\frac{d^nx}{dt^n}=x^{(n)}(t)=x^{(n)}$.

I dette projekt tages der udgangspunkt i Leibniz' notation. Der vil dog forekomme undtagelser, hvor Lagranges notation vil blive anvendt, men dette skift i notation vil blive noteret i den tilhørende tekst.

Med udgangspunkt i Leibniz' notation defineres en $n$'te differentialligning.

\begin{defn}\label{def:generel_differentialligning} \textbf{}\\
Lad $D_f \subseteq \R \times \R^n$ være en åben ikke-tom delmængde og $f \in C^0(D_f,\R)$. En $n$'te ordens differentialligning kan da udtrykkes som følgende
%
\begin{align*}
    \frac{d^nx}{dt^n}=f\left(t, x, \frac{dx}{dt},\cdots, \frac{d^{n-1}x}{dt^{n-1}}\right )
\end{align*}
hvor $t \in I$ og $x \in C^n(I \to \R)$. 
\end{defn}

$C^n(D_f, V_f)$ er mængden af alle kontinuerte differentiable funktioner af $n$'te orden, hvor $n \geq 0$. For en funktion $f \in C^n(D_f, V_f)$ gælder det, at $f: D_f \to V_f$. 


