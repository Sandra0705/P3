\subsection{Linearitet for differentialligninger}
Dette afsnit er baseret på \cite{"Linaere_foerste_ordens"}. 

Inden \emph{lineære differentialligninger} introduceres, skal linearitet for disse defineres.
\newline
Hvis der findes en vilkårlig differentialligning
\begin{equation*}
    q(t)=f(x), \quad t \in I
\end{equation*}
hvor $q \in C^0(I, \R)=V_f$ er en funktion, der afhænger af $t$, og $x \in C^n(I, \R)=D_f$ er den ukendte funktion. $f \in C^0(D_f, V_f)$, hvor $f: \R \rightarrow \R$ er en lineær afbildning. Når $f$ er en lineær afbildning, vil differentialligningen blive kaldt lineær.\\

\begin{defn}\textbf{Lineære differentialligninger}
\newline
En differentialligning $f(x) = q(t)$ er lineær, hvis funktionen $f(x)$ opfylder begge linearitetsbetingelser
\begin{itemize}
    \item[1.] $f(x_1+x_2)=f(x_1)+f(x_2)$
    \item[2.]$f(c\cdot x) = c \cdot f(x)$,
\end{itemize}
altså hvis $f \in C^0 (D_f, V_f)$ er en lineær afbildning. $x, x_1, x_2$ er løsninger til differentialligningen, og $c \in \R$ er en skalar.
\end{defn}

% \begin{thmx}\textbf{Løsningsstruktur for lineære differentialligninger}
% \newline
% Den lineære differentialligning
% \begin{equation*}
%     f\left(x(t)\right)=q(t), \quad t \in I
% \end{equation*}
% kaldes \textit{inhomogen}, når $q(t) \neq 0$. Tilsvarende vil differentialligningen være \textit{homogen}, når $q(t)=0$.
% \newline
% \textit{Den fuldstændige løsningmængde} $L_{inhomogen}$ til den inhomogene differentialligningen har formen
% \begin{equation*}
%     L_{inhomogen} = x_p(t)+L_{homogen},
% \end{equation*}
% hvor $x_p(t)$ er en \textit{partikulær løsning} til den inhomogene differentialligning, og $L_{homogene}$  er den fuldstændige løsning til den tilsvarende homogene differentialligning.
% \end{thmx}
% \begin{bev}
% ja det gælder
% \end{bev}
% \begin{eks}
% Eksempel på homogen
% \end{eks}


% \begin{defn}\textbf{Lineær første ordens differentialligninger}\label{Def_Lineær_Første_Orndens_Differential_Ligninger}
%  \newline
%  En første ordens sædvanlig differential ligning på et ukendt y-plan er
%  \begin{align}\label{Eq:Lineær_Første_Ordens_Differential_Ligninger1}
%      y'(t) = f(t,y(t)),
%  \end{align}
%  hvor \textit{f} er givet og \textit{y'} = $\frac{dy}{dt}$. Ligningen \eqref{Eq:Lineær_Første_Ordens_Differential_Ligninger} er lineær hvis \textit{f} er linear ved sit andet argument
% \begin{align}\label{Eq:Lineær_Første_Ordens_Differential_Ligninger2}
%     y´= a(t)y + b(t).
% \end{align}
% Den lineære ligning har konstante koefficiener, hvis både \textit{a} og \textit{b} er konstanter. Ellers har den lineære ligning variable koefficienter.
% \end{defn}



