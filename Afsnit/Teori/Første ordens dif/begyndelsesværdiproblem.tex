\subsubsection{Begyndelsesværdiproblem}

Dette afsnit er baseret på \cite{"Moller"}

% Et begyndelsesværdiproblem har to betingelser:
% \begin{itemize}
%     \item En første ordens differentialligning $\frac{dy}{dx}=f(x,y)$.
%     \item Begyndelsesbetingelse på formen $y(a)=b$.
% \end{itemize}

% Løsning af begyndelsesværdiproblem:

\begin{defn}\textbf{Begyndelsesværdiproblemet} %Ny definition
\newline
Lad $D_f \subseteq \R \times \R, f: D_f \rightarrow \R$ og $(t_0, x_0) \in D_f$. Så er begyndelsesværdiproblemet givet ved 
\begin{align*}
    \frac{dx}{dt} = f(t, x), \quad x(t_0) = x_0
\end{align*}
hvor $x(t_0) = x_0$ betegner \textit{begyndelsesbetingelsen}
\end{defn}

Senere i projektet skal løsningen for begyndelsesværdiproblemet bruges for en homogen første ordens differentialligning med konstante koefficienter. Derfor vil denne blive gennemgået. 

\begin{thmx}\textbf{En løsning til begyndelsesværdiproblemet}\\
Lad \textit{a} være en konstant og lad $\frac{dx}{dt}: \R \rightarrow \R$. Da vil begyndelsesværdiproblemet 
    \begin{align}\label{begyndelsesværdi_konstant_a}
        \frac{dx}{dt}=ax, \quad x(t_0)=x_0
    \end{align}
    have løsningen 
    \begin{align*}
        x(t) = e^{(t-t_0)a}x_0
    \end{align*}
\end{thmx}

\begin{bev}\textbf{}\\
    Det vides fra tidligere, at \eqref{eq:løsning_til_homo_første_ordens} er den generelle løsning til $\displaystyle \frac{dx}{dt}=ax$.
    
    Begyndelsesværdiproblemet indsættes i \eqref{eq:løsning_til_homo_første_ordens}.
    \begin{align*}
        x_0 = c\cdot e^{t_0a}\\
        \intertext{$c$ isoleres.}
        c=x_0 \cdot e^{-t_0a}\\
        \intertext{$c$ indsættes i \eqref{eq:løsning_til_homo_første_ordens}}
        x(t)=x_0\cdot e^{-t_0a}e^{ta}\\
        x(t)=x_0\cdot e^{(t-t_0)a}
    \end{align*}
\end{bev}


\begin{eks} \textbf{} %Nyt eksempel
\newline
Lad begyndelsesværdiproblemet være givet ved differentialligningen $\frac{dx}{dt} = -2x+1$ og begyndelsesbetingelsen $x(0)=2$. 

Dette er en inhomogen første ordens differentialligning med konstante koefficienter, og derfor kan den generelle løsning bestemmes ud fra \eqref{eq:løsning_til_inhomo_første_ordens}.
%
\begin{align*}
    x = ce^{t(-2)} + \frac{1}{2}
\end{align*}
Herefter bestemmes $c$ ved at indsætte begyndelsesbetingelsen $x(0) = 2$
%
\begin{align*}
    2 = ce^{0\cdot(-2)} - \frac{1}{2} = c - \frac{1}{2} \implies c = \frac{5}{2}
\end{align*}
%
Den partikulære løsning til $\frac{dx}{dt} = -2x+1$ er $x=\frac{5}{2}e^{t(-2)} + \frac{1}{2}$ for begyndelsesbetingelsen $x(0)=2$.
\end{eks}