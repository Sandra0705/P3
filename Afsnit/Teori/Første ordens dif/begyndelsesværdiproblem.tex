\subsubsection{Begyndelsesværdiproblem}

Dette afsnit er baseret på \cite{"Moller"}

% Et begyndelsesværdiproblem har to betingelser:
% \begin{itemize}
%     \item En første ordens differentialligning $\frac{dy}{dx}=f(x,y)$.
%     \item Begyndelsesbetingelse på formen $y(a)=b$.
% \end{itemize}

% Løsning af begyndelsesværdiproblem:

\begin{defn}\textbf{} %Ny definition
\newline
Lad $D_f \subseteq \R \times \R^n, f: D_f \rightarrow \R^n$ og $(t_0, x_0) \in D_f$. Så er begyndelsesværdi-problemet givet ved 
\begin{align*}
    \frac{dx}{dt} = f(t, x), \quad x(t_0) = x_0
\end{align*}
Hvor $x(t_0) = x_0$ betegner begyndelsesbetingelsen
\end{defn}

\begin{eks} \textbf{} %Nyt eksempel
\newline
Lad differentialligningen $\frac{dx}{dt} = -2x(t)+1$ og begyndelsesværdiproblemet $x(0)=2$ være givet. 

Dette er en inhomogen første ordens differentialligning med konstante koefficienter. Derfor kan den generelle løsning bestemmes ud fra \eqref{eq:løsning_til_inhomo_første_ordens}.
%
\begin{align*}
    x(t) = ce^{-2t} + \frac{1}{2}
\end{align*}
Herefter bestemmes c ved at indsætte begyndelsesbetingelsen $x(0) = 2$
%
\begin{align*}
    2 = ce^{-2\cdot0} - \frac{1}{2} = c - \frac{1}{2} \implies c = \frac{5}{2}
\end{align*}
%
Den partikulære løsning til $\frac{dx}{dt} = -2x(t)+1$ er $x(t)=\frac{5}{2}e^{-2t} + \frac{1}{2}$ for begyndelsesbetingelsen $x(0)=2$.
\end{eks}

