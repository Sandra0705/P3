\subsection{Løsninger til første ordens differentialligninger}
Dette afsnit er baseret på \cite[s. 15-16]{"ODE"}.

Den generelle løsning til en lineær første ordens differentialligning kan bestemmes med følgende sætning. 

\begin{thmx} \textbf{Generel løsning til en lineær første orden differentialligning} %Ny sætning
\newline
Lad $a(t), b(t)$ være kontinuerte funktioner og lad $\frac{dx}{dt}:\R \xrightarrow{} \R$. 
Da vil den lineære differentialligning
\begin{align}
    \frac{dx}{dt}+a(t)x=+b(t)
\end{align}
have den generelle løsning,
\begin{align}
    x(t)=ce^{-A(t)} + e^{-A(t)} \int e^{A(t)}b(t) \ dt
\end{align}
hvor $A(t)=\int a(t) \ dt$ og $c \in \R$.

\end{thmx}
%
\begin{bev} \textbf{} %Nyt bevis
\newline

Lad en lineær første ordens differentialligning være givet ved følgende
\begin{align}
    \frac{dx}{dt} + a(t)x &= b(t) \nonumber\\
    \intertext{Det vides at $a(t)$ er kontinuert, og der må derfor findes en stamfunktion $A(t)$. Der ganges med $e^{A(t)}$ på begge sider af lighedstegnet}
    \left(\frac{dx}{dt} + a(t)x\right) \cdot e^{A(t)} &= b(t)\cdot e^{A(t)} \nonumber\\
    \intertext{$e^{A(t)}$ bliver nu ganget ind i parenteserne}
    \frac{dx}{dt}\cdot e^{A(t)} + a(t)x\cdot e^{A(t)}&=b(t)\cdot e^{A(t)} \label{panser_produkt} \\
    \intertext{Da \eqref{panser_produkt} er på formen $\frac{df}\cdot{dt}g(t) + f(t)\frac{dg}\cdot{dt}$ anvendes produktreglen}
    \frac{d}{dt}(e^{A(t)} \cdot x) &= b(t)\cdot e^{A(t)} \nonumber\\
    \intertext{Der integreres på begge sider af lighedstegnet}
    \int \frac{d}{dt}(e^{A(t)} \cdot x )\ dt &= \int b(t) \  \cdot e^{A(t)} \ dt \nonumber \\
    \intertext{På venstresiden af udtrykket udligner differentiation og integration hinanden. Da højresiden også integreres dannes der et konstantled c.}
    e^{A(t)} \cdot x &= \int b(t) \cdot e^{A(t)} \ dt + c \nonumber \\
    \intertext{Både højresiden og venstresiden ganges med $e^{-A(t)}$ for at isolere $x$}
    x &= e^{-A(t)} \cdot \left(\int e^{A(t)} \cdot b(t) \ dt + c\right) \nonumber\\
    \intertext{$e^{-A(t)}$ ganges ind i parantesen}
    x &= ce^{-A(t)} +e^{-A(t)} \cdot \int e^{A(t)} \cdot b(t) \ dt \nonumber
\end{align}
\end{bev}
Differentialligningnen kaldes for homogen hvis $b(t) = 0$ og hvis $b(t) \neq 0$ kaldes den for en inhomogen differentialligning. 

For en lineær første ordens differentialligning med konstante koefficienter
\begin{align}\label{eq:første_ordens_differential_ligning_med_konstante_koefficienter}
    \frac{dx}{dt}+ax=b \text{ for } a,b \in \R
\end{align}
vil den generelle løsning kunne udtrykkes ved
\begin{align} \label{eq:løsning_til_inhomo_første_ordens}
    x(t)=ce^{-at}+\frac{b}{a}
\end{align}
Hvis $b=0$, kan den generelle løsning kunne reduceres til 
\begin{align}\label{eq:løsning_til_homo_første_ordens}
    x(t) = ce^{-at}
\end{align}

Disse er de generelle løsninger til en første ordens differentialligning. Hertil findes der uendeligt mange partikulære løsninger, som kan bestemmes ud fra begyndelsesværdiproblemet. 


