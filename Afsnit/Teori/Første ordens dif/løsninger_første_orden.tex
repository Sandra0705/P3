\subsection{Løsninger til første ordens differentialligninger}
Dette afsnit er baseret på \cite[s. 15-16]{"ODE"}.

Den generelle løsning til en lineær første ordens differentialligning kan bestemmes med følgende sætning. 

\begin{minipage}\textwidth
\begin{thmx} \textbf{Generel løsning til en lineær første orden differentialligning} %Ny sætning
\newline
Lad $a(t)$ og $b(t)$ være kontinuerte funktioner og lad $x:I \xrightarrow{} \R$. 
Da vil den lineære differentialligning
\begin{align}
    \frac{dx}{dt}=a(t)x+b(t)
\end{align}
have den generelle løsning,
\begin{align}\label{eq:generel_løsning_til_1.ordens_med_variable_koefficienter}
    x(t)=ce^{A(t)} + e^{A(t)} \int e^{-A(t)}b(t) \ dt
\end{align}
hvor $\displaystyle A(t)=\int a(t) \ dt$ og $c \in \R$.
\end{thmx}
\end{minipage}
%
\begin{bev} \textbf{} %Nyt bevis
\newline
Lad en lineær første ordens differentialligning være givet ved følgende
\begin{align}
    \frac{dx}{dt} &= a(t)x + b(t) \nonumber\\
    \intertext{Det vides, at $a(t)$ er kontinuert, og der må derfor findes en stamfunktion $A(t)$. Der multipliceres med $e^{-A(t)}$ på begge sider af lighedstegnet}
    \frac{dx}{dt}  \cdot e^{-A(t)} &=\left(a(t)x + b(t)\right)\cdot e^{-A(t)} \nonumber\\
    \intertext{$e^{-A(t)}$ bliver nu multipliceret ind i parenteserne}
    \frac{dx}{dt}\cdot e^{-A(t)} &= a(t)x\cdot e^{-A(t)} + b(t)\cdot e^{-A(t)}\nonumber \\
    \intertext{$a(t)x\cdot e^{-A(t)}$ subtraheres på begge sider}
    \frac{dx}{dt}\cdot e^{-A(t)} - a(t)x\cdot e^{-A(t)} &= b(t)\cdot e^{-A(t)} \label{panser_produkt}\\
    \intertext{Da \eqref{panser_produkt} er på formen $\frac{df}{dt}g(t) + f(t)\frac{dg}{dt}$, anvendes produktreglen for differentiation} 
    \frac{d}{dt}(e^{-A(t)} \cdot x) &= b(t)\cdot e^{-A(t)} \nonumber\\
    \intertext{Der integreres på begge sider af lighedstegnet}
    \int \frac{d}{dt}(e^{-A(t)} \cdot x )\ dt &= \int b(t) \  \cdot e^{-A(t)} \ dt \nonumber \\
    \intertext{På venstresiden af udtrykket reducerer differentiation og integration hinanden. På grund af integration, dannes der et konstantled på begge sider af lighedstegnet. De konstante led samles på højresiden og betegnes med $c$.}
    e^{-A(t)} \cdot x &= \int b(t) \cdot e^{-A(t)} \ dt + c \nonumber \\
    \intertext{Både højresiden og venstresiden multipliceres med $e^{A(t)}$ for at isolere $x$.}
    x(t) &= e^{A(t)} \cdot \left(\int e^{-A(t)} \cdot b(t) \ dt + c\right) \nonumber\\
    \intertext{$e^{A(t)}$ multipliceres ind i parentesen}
    x(t) &= ce^{A(t)} +e^{A(t)} \cdot \int e^{-A(t)} \cdot b(t) \ dt \nonumber
\end{align}
\end{bev}
Differentialligningnen kaldes for homogen, hvis $b(t) = 0$, og hvis $b(t) \neq 0$ kaldes den for en inhomogen differentialligning. 

For en lineær første ordens differentialligning med konstante koefficienter
\begin{align}\label{eq:første_ordens_differential_ligning_med_konstante_koefficienter}
    \frac{dx}{dt}=ax+b \text{ for } a,b \in \R
\end{align}
vil den generelle løsning, ud fra \eqref{eq:generel_løsning_til_1.ordens_med_variable_koefficienter}, kunne udtrykkes ved
\begin{align} \label{eq:løsning_til_inhomo_første_ordens}
    x(t)=ce^{ta}+\frac{b}{a}
\end{align}
Hvis $b=0$, kan den generelle løsning reduceres til 
\begin{align}\label{eq:løsning_til_homo_første_ordens}
    x(t) = ce^{ta}
\end{align}

Disse er de generelle løsninger til en første ordens differentialligning. Hertil findes der uendeligt mange partikulære løsninger, som kan bestemmes ud fra \textit{begyndelsesværdiproblemet}. 


