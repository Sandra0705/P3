Der vil blive opstillet et udtryk for $e^{tA}$, når $tr(A) \neq 0$. 

\begin{minipage}\textwidth
\begin{thmx} \textbf{}\label{sæt:løsning_2_x_2_ved_egenværdier_etA_beregning} %Ny sætning
\newline
Lad $A\in \R^{2\times 2}$, og lad $B = A - \frac{1}{2}\left(tr(A)\right)I_2$. Så gælder det, at hvis 
\begin{enumerate}
    \item $A$ har én reel egenværdi, så er $$e^{tA}=e^{t\lambda}(I_2+tB)$$
    \item $A$ har to forskellige reelle egenværdier, så er $$e^{tA}=e^{t\frac{1}{2}(\lambda_1+\lambda_2)}\left(cosh\left(t\sqrt{-det(B)}\right)I_2+\frac{sinh\left(t\sqrt{-det(B)}\right)}{\sqrt{-det(B)}}B\right)$$
    \item $A$ har et par af kompleks konjugerede egenværdier, så er $$e^{tA}=e^{t\frac{1}{2}(\lambda_1+\lambda_2)}\left(cos\left(t\sqrt{det(B)}\right)I_2+\frac{sin\left(t\sqrt{det(B)}\right)}{\sqrt{det(B)}}B\right)$$
\end{enumerate}  
\end{thmx}
\end{minipage}

\begin{bev} \textbf{} %Nyt bevis
\newline
Lad $A\in \R^{2\times 2}$, og lad $B = A - \frac{1}{2}(tr(A))I_2$. 

Først opstilles $e^{tA}$ ved brug af $B$.
Fra \autoref{sæt:eta_med_trace} kan $e^{tA}$ udtrykkes som $$e^{tA} = e^{t\left(\frac{1}{2}tr(A)\right)}e^{t\left(A-\frac{1}{2}tr(A)\right)I_2}$$ $B$ indsættes i udtrykket.
    \begin{align}
        e^{tA} &= e^{t\left(\frac{1}{2}{tr}(A)\right)}e^{tB} \label{eq:eta_indsat_B}
     \end{align}
     %
Ved at opstille det karakteristiske polynomium, er det muligt at udtrykke $tr(A)$ og $det(A)$ ved brug af egenværdierne til $A$. Det karakteristiske polynomium opstilles, se \autoref{bilag:det_karakteristiske_polynomium}.
%
\begin{align}
    p_A(z) &= (-1)^2(z-\lambda_1)(z-\lambda_2)\nonumber \\
    &= z^2 - (\lambda_1+\lambda_2)z + \lambda_1 \lambda_2\nonumber
\intertext{Fra \eqref{eq:det_karakteristiske_polynomium_udtrykt_ved_sporet} ses det, at $tr(A)=\lambda_1+\lambda_2$, og $det(A)=\lambda_1\lambda_2$.}
\intertext{$tr(A)=\lambda_1+\lambda_2$ indsættes i \eqref{eq:eta_indsat_B}.}
     e^{tA} &= e^{t\left(\frac{1}{2}(\lambda_1+\lambda_2)\right)}e^{tB} \label{Den_vi_skal_bruge_til_at_bevise_6.0.1}
\end{align}
Dernæst bestemmes et udtryk for determinanten af $B$. 
%Egenværdierne for $B$ kan opskrives ud fra \autoref{sæt:egenværdier_og_spor}, da \autoref{sæt:tr(a-led)=0} viser, at matricen opfylder kravene til \autoref{sæt:egenværdier_og_spor}. 
% 
\begin{align*}
    det(B) &= det(A)-det\left(\frac{1}{2}tr(A)I_2\right) \nonumber \\
    &= det(A) - \left(\frac{1}{2}tr(A)\right)\cdot\left(\frac{1}{2}tr(A)\right) - 0\cdot0 \\
    &=det(A)-\frac{1}{4}\left(tr(A)\right)^2
\intertext{Herefter kan egenværdierne $\lambda_1$ og $\lambda_2$ indsættes i ovenstående udtryk.}
    det(B) &=\lambda_1\lambda_2 - \frac{1}{4} \left(\lambda_1 + \lambda_2\right) 
\end{align*}
Dermed er det muligt at anvende \autoref{e^tA_udtrykt_når_trA=0} til at bestemme et udtryk for $e^{tB}$.

\begin{itemize}
    \item [] \textbf{Bevis for punkt 1}\\
    Antag, at $A$ kun har én reel egenværdi, $\lambda_1,\lambda_2\in\R$, hvor $\lambda_1 = \lambda_2$.
    
    $det(B)$ bestemmes.
    \begin{align*}
    det(B) &= \lambda^2-\frac{1}{4}\left(2\lambda\right)^2 \\
        &= \lambda^2-\lambda^2 \\
        &= 0
    \intertext{Da $\lambda_1=\lambda_2$, gælder følgende for \eqref{Den_vi_skal_bruge_til_at_bevise_6.0.1}}
    e^{tA} &= e^{t\lambda}e^{tB}
    \intertext{Da $det(B)=0$ anvendes \autoref{e^tA_udtrykt_når_trA=0} punkt 1 til at omskrive $e^{tB}$.}
    %\intertext{Udtrykket for $e^{tB}$, \autoref{e^tA_udtrykt_når_trA=0}, for $det(B)=0$ indsættes.}
    e^{tA} &= e^{t\lambda}(I_2+tB)
    \end{align*}
    Altså er det bevist, at $e^{tA} = e^{t\lambda}(I_2+tB)$, når $A$ har én egenværdi. 
    %
     \item[] \textbf{Bevis for punkt 2}\\
     Antag, at $A$ har to forskellige reelle egenværdier, $\lambda_1\neq\lambda_2\in\R$.
     
    $det(B)$ bestemmes.
    \begin{align*}
        det(B) &= \lambda_1\lambda_2-\frac{1}{4}(\lambda_1+\lambda_2)^2 \\
        %&= \lambda_1\lambda_2-\frac{1}{4}(\lambda_1^2+\lambda_2^2+2\lambda_1\lambda_2) \\
        &= \lambda_1\lambda_2-\frac{1}{4}\lambda_1^2-\frac{1}{4}\lambda_2^2-\frac{1}{2}\lambda_1\lambda_2 \\
        &= -\frac{1}{4}\lambda_1^2-\frac{1}{4}\lambda_2^2 + \frac{1}{2}\lambda_1\lambda_2 \\
        &= -\left(\frac{1}{2}\lambda_1-\frac{1}{2}\lambda_2\right)^2
    \end{align*}
    Da $\lambda_1 \neq \lambda_2$, medfører det, at $det(B)<0$.
    
    Da $det(B)<0$ anvendes \autoref{e^tA_udtrykt_når_trA=0} punkt 2 til at omskrive $e^{tB}$.
    %Udtrykket for $e^{tB}$, \autoref{e^tA_udtrykt_når_trA=0}, for $det(B)<0$ indsættes i \eqref{Den_vi_skal_bruge_til_at_bevise_6.0.1}
    \begin{align*}
     e^{tA} = e^{t\left(\frac{1}{2}(\lambda_1+\lambda_2)\right)}\left( \cosh\left(t\sqrt{-det(B)}\right)I_2+\frac{\sinh\left(t\sqrt{-det(B)}\right)}{\sqrt{-det(B)}}B\right)
    \end{align*}
    Altså er det bevist, at $e^{tA}=e^{t\left(\frac{1}{2}(\lambda_1+\lambda_2)\right)}\left( \cosh\left(t\sqrt{-det(B)}\right)I_2+\frac{\sinh\left(t\sqrt{-det(B)}\right)}{\sqrt{-det(B)}}B  \right)$, når $A$ har to forskellige reelle egenværdier. 
    %
    \item [] \textbf{Bevis for punkt 3}\\
    Antag, at $A$ har et par af kompleks konjugerede egenværdier, $\lambda_1\neq\lambda_2\in\C\backslash\R$. \\
    Egenværdierne udtrykkes som $\lambda_1 = a+bi$ og $\lambda_2 = a - bi$.
     
    $det(B)$ bestemmes. 
    \begin{align*}
        det(B) &=\left(\left(a+bi\right)\left(a-bi\right)\right) - \frac{1}{4}\left((a+bi)+(a-bi)\right)^2\\
        &= a^2+b^2-\frac{1}{4}4a^2 \\
        %&= a^2+b^2-\frac{1}{4}\left(2a\right)^2 \\
        %&= a^2+b^2-\frac{1}{4}4a^2 \\
        &= b^2
    \end{align*} 
    %
    Da $\lambda_1,\lambda_2\in \C\backslash\R$, vil $b^2>0$, hvoraf $det(B)>0$.
    %
    \begin{align*}
     \intertext{Da $det(B)>0$, anvendes \autoref{e^tA_udtrykt_når_trA=0} punkt 3 til at omskrive $e^{tB}$.}
     e^{tA} &= e^{t\left(\frac{1}{2}(\lambda_1+\lambda_2)\right)} \left(cos\left(t\sqrt{det(A)}\right)I_2+\frac{\sin\left(t\sqrt{det(A)}\right)}{\sqrt{det(A)}}A\right)
     \end{align*}
    Altså er det bevist, at $e^{tA} = e^{t\left(\frac{1}{2}(\lambda_1+\lambda_2)\right)} \left(cos\left(t\sqrt{det(A)}\right)I_2+\frac{\sin\left(t\sqrt{det(A)}\right)}{\sqrt{det(A)}}A\right)$, når $A$ har to kompleks konjugerede egenværdier. 
\end{itemize}
\end{bev}