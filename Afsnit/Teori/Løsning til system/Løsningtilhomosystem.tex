\section{Løsninger til lineære systemer}

Da der for enhver matrice $A$ er defineret en eksponentialmatrice $e^{tA}$, er det muligt at udlede løsninger til autonome lineære systemer af første ordens differentialligninger.
%Ud fra løsningen til begyndelsesværdiproblemet for autonome lineære differentialligninger, vil der nu opstilles løsninger for autonome lineære systemer. 

\begin{minipage}\textwidth
\begin{thmx} \textbf{Løsning til et autonomt lineært system af homogene første ordens differentialligninger} \label{sæt:løsning_til_homogen_system_af_første_ordens_differentialligninger}%Ny sætning
\newline
Lad $A \in \R^{n \times n}$, $t\in I$ og $\textbf{x}: I \to \R^n$. Lad $D_\textbf{f} \subseteq \R^n$ være en åben ikke-tom delmængde og $\textbf{f} \in C^1(D_\textbf{f},\R^n)$. Da gælder det, at begyndelsesværdiproblemet for det homogene autonome lineære system 
\begin{align}\label{eq:homogen_lineart_system}
    \frac{d\textbf{x}}{dt} = A\textbf{x}, \quad \mathbf{x}(t_0) = \mathbf{x}_0 
\end{align}
har den entydig bestemte løsning 
\begin{align}\label{eq:løsning_til_homogen_lineart_system}
    \textbf{x}(t) = e^{(t-t_0)A}\textbf{x}_0
\end{align}
\end{thmx}
\end{minipage}

\begin{bev}\textbf{}\\
Det vil blive bevist, at \eqref{eq:løsning_til_homogen_lineart_system} er en løsning til \eqref{eq:homogen_lineart_system}.
%
    \begin{align*}
    \textbf{x}(t) &= e^{(t-t_0)A}\textbf{x}_0\\
     &= e^{tA}e^{-t_0A}\textbf{x}_0
    \intertext{$\textbf{x}(t)$ differentieres med hensyn til $t$.}
    \frac{d\textbf{x}}{dt} &= Ae^{tA}e^{-t_0A}\textbf{x}_0\\ 
    &= Ae^{(t-t_0)A}\textbf{x}_0
    \intertext{Da $\textbf{x}(t)=e^{(t-t_0)A}\textbf{x}_0$, indsættes $\textbf{x}$ i udtrykket.}
    \frac{d\textbf{x}}{dt} &= A\textbf{x}
    \end{align*}
Dermed er det bevist, at \eqref{eq:løsning_til_homogen_lineart_system} er en løsning til \eqref{eq:homogen_lineart_system}.
\end{bev}

Ved brug af ovenstående resultat, kan løsningen til begyndelsesværdiproblemet for et inhomogent system også bestemmes. 

\begin{minipage}\textwidth
\begin{thmx} \textbf{Løsning til et lineært system af inhomogene første ordens differentialligninger} \label{sæt:løsning_til_inhomogen_system}%Ny sætning
\newline
Lad $A \in \R^{n \times n}$, $t\in I$ og $\textbf{x}: I \to \R^n$. Lad $D_\textbf{f} \subseteq \R\times \R^n$ være en åben ikke-tom delmængde og $\textbf{f} \in C^1(D_\textbf{f}, \R^n)$. Lad derudover $\mathbf{b}(t): \R \rightarrow \R^n$ være en given kontinuert funktion. Da gælder det, at begyndelsesværdiproblemet for det inhomogene lineære system 
\begin{align}\label{eq:inhomogen_lineart_system}
    \frac{d\textbf{x}}{dt} = A\textbf{x}+\textbf{b}(t), \quad \mathbf{x}(t_0) = \mathbf{x}_0 
\end{align}
har den entydigt bestemte løsning 
\begin{align}\label{eq:løsning_til_inhomogen_lineart_system}
    \textbf{x}(t) = e^{(t-t_0)A}\textbf{x}_0 + \int_{t_0}^t e^{(t-s)A}\textbf{b}(s) ds
\end{align}
\end{thmx}
\end{minipage}

\begin{bev} \textbf{} %Nyt bevis
\newline
Først vil det blive bevist, at løsningen opfylder kravene for eksistens- og entydighedssætningen (se \autoref{sæt:eksistens_og_entydighed}). 

Det er defineret, at $A\textbf{x}(t)$ og $\textbf{b}(t)$ er kontinuerte funktioner, da de er lineære. Det vides, at for
$\displaystyle \frac{d\textbf{x}}{dt}=\textbf{f}(t,\textbf{x})$ gælder det, at $D_\textbf{f} \subseteq \R \times \R^n$ er en ikke-tom delmængde, og $\textbf{f}: D_\textbf{f} \to \R^n$. Da $\textbf{f}(t,\textbf{x})$ er defineret på $\R \times \R^{n}$, er $D_\textbf{f}$ et åbent interval. Funktionen opfylder altså kravene for eksistens- og entydighedssætningen (se \autoref{sæt:eksistens_og_entydighed}).

Lad $\textbf{y}_1$ og $\textbf{y}_2$ være løsninger til \eqref{eq:inhomogen_lineart_system}.
Løsningerne indsættes i \eqref{eq:Eksistens_og_entydighed}.
%
\begin{align*}
    \norm{\textbf{f}(t, \textbf{y}_1)-\textbf{f}(t, \textbf{y}_2)}
    &=\norm{A\textbf{y}_1+\textbf{b}(t)-\left(A\textbf{y}_2+\textbf{b}(t)\right)}\\
    \intertext{Højresiden reduceres.}
    \norm{A\textbf{y}_1+\textbf{b}(t)-\left(A\textbf{y}_2+\textbf{b}(t)\right)}&=\norm{A(\textbf{y}_1-\textbf{y}_2)}\\
    \intertext{\autoref{sæt:frobenius_egenskab_1} anvendes.}
    \norm{A(\textbf{y}_1-\textbf{y}_2)}&\leq\norm A _F \norm{\textbf{y}_1-\textbf{y}_2}\\
    &\Updownarrow\\
    \norm{\textbf{f}(t, \textbf{y}_1)-\textbf{f}(t, \textbf{y}_2)}&\leq \norm A _F \norm{\textbf{y}_1-\textbf{y}_2}
\end{align*}
Det er dermed bevist, ud fra eksistens- og entydighedssætningen (se \autoref{sæt:eksistens_og_entydighed}), at \eqref{eq:inhomogen_lineart_system} har en entydig løsning.

Dernæst vil den entydige løsning blive bestemt.
%
\begin{align}\label{eq:løsningen_til_inhomogen_system_inden_begyndelsesbetingelsen_er_indsat}
    \frac{d\textbf{x}}{dt} &= A\textbf{x} + \textbf{b}(t) \nonumber
    \intertext{Der subtraheres med $A\textbf{x}$ på begge sider.}
    \frac{d\textbf{x}}{dt}-A\mathbf{x}&=\mathbf{b}(t)\nonumber\\
    \intertext{Der multipliceres med $e^{-tA}$ på begge sider.}
    e^{-tA}\left(\frac{d\textbf{x}}{dt}-A\mathbf{x}\right) &= e^{-tA}\mathbf{b}(t)\nonumber\\
    &\Updownarrow\nonumber\\
    e^{-tA}\frac{d\textbf{x}}{dt}-e^{-tA}A\mathbf{x}&=e^{-tA}\mathbf{b}(t)\nonumber\\
    \intertext{Da ovenstående er på formen $\frac{df}{dt}g(t) + f(t)\frac{dg}{dt}$, anvendes produktreglen for differentiation.}
    \frac{d}{dt}\left(e^{-tA}\mathbf{x}\right)&=e^{-tA}\mathbf{b}(t) \nonumber\\
    \intertext{Begge sider af udtrykket integreres.}
    \int \frac{d}{dt}\left(e^{-tA}\mathbf{x}\right) dt&= \int
    e^{-tA}\mathbf{b}(t) dt\nonumber\\
    \intertext{På venstresiden af udtrykket reducerer differentiation og integration hinanden. På grund af integration, dannes der et konstantled på begge sider af lighedstegnet. De konstante led samles på højresiden og betegnes $c$.}
    e^{-tA}\mathbf{x} &= \int e^{-tA}\mathbf{b}(t) dt + c\nonumber\\
    \intertext{Der multipliceres med $e^{tA}$ på begge sider af lighedstegnet for at isolere \textbf{x}.}
    \mathbf{x}(t) &= e^{tA}c + e^{tA}\cdot \int e^{-tA}\mathbf{b}(t) dt
\end{align}

Begyndelsesbetingelsen, $\textbf{x}(t_0)=\textbf{x}_0$, indsættes i \eqref{eq:løsningen_til_inhomogen_system_inden_begyndelsesbetingelsen_er_indsat}, og $c$ isoleres.
\begin{align*}
    \mathbf{x}_0 &= e^{t_0A}c + e^{t_0A}\cdot \int e^{-t_0A}\mathbf{b}(t_0) dt\\
    &\Uptown\\
    e^{t_0A}c &= \textbf{x}_0 - e^{t_0A}\cdot \int e^{-t_0A}\mathbf{b}(t_0) dt\\
    &\Uptown\\
    c &= e^{-t_0A}\textbf{x}_0 - \int e^{-t_0A}\mathbf{b}(t_0) dt
    \intertext{Udtrykket for $c$ indsættes i \eqref{eq:løsningen_til_inhomogen_system_inden_begyndelsesbetingelsen_er_indsat}.}
    \textbf{x}(t) &= e^{tA} \left( e^{-t_0A}\textbf{x}_0 - \int e^{-t_0A}\mathbf{b}(t_0) dt \right) + e^{tA}\cdot \int e^{-tA}\mathbf{b}(t) dt
    \intertext{$e^{tA}$ multipliceres ind i parentesen.}
    \textbf{x}(t) &= e^{tA}e^{-t_0A}\textbf{x}_0 - e^{tA} \cdot \int e^{-t_0A}\mathbf{b}(t_0) dt + e^{tA}\cdot \int e^{-tA}\mathbf{b}(t) dt\\
    \intertext{\autoref{sæt:egenskaber_for_eta} punkt 3 anvendes, og $e^{tA}$ faktoriseres ud.}
    \textbf{x}(t) &= e^{(t-t_0)A}\textbf{x}_0 + e^{tA} \cdot \left( \int e^{-tA}\mathbf{b}(t) dt -  \int e^{-t_0A}\mathbf{b}(t_0) dt \right)
    \intertext{De ubestemte integraler omskrives til det bestemte integral fra $t_0$ til $t$ af $s$.}
    \textbf{x}(t) &= e^{(t-t_0)A}\textbf{x}_0+ e^{tA} \int_{t_0}^t e^{-sA}\mathbf{b}(s) ds\\
    \intertext{$e^{tA}$ multipliceres ind i integralet, og \autoref{sæt:egenskaber_for_eta} punkt 3 anvendes igen.}
    \textbf{x}(t) &= e^{(t-t_0)A}\textbf{x}_0 + \int_{t_0}^t e^{tA} e^{-sA}\mathbf{b}(s) ds\\
     &= e^{(t-t_0)A}\textbf{x}_0 + \int_{t_0}^t e^{tA-sA}\mathbf{b}(s) ds\\
    &= e^{(t-t_0)A}\textbf{x}_0 + \int_{t_0}^t e^{(t-s)A}\mathbf{b}(s) ds
\end{align*}
Dermed er det bevist, at \eqref{eq:løsning_til_inhomogen_lineart_system} er den entydigt bestemte løsning til \eqref{eq:inhomogen_lineart_system}.

\end{bev}



