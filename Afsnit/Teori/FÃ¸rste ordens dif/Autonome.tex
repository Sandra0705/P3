\subsection{Autonome første ordens differentialligninger}

Dette afsnit er baseret på \cite{"ODE"}.

En differentialligning, som ikke eksplicit afhænger af $t$, men kun $x$, kaldes en \textit{autonom differentialligning}.

\begin{defn} \textbf{Autonome første ordens differentialligninger} 
\newline 
Lad $t \in I$ og $x: I \to \R$. Lad $D_f \subseteq \R$ være en åben ikke-tom delmængde hvor $f \in C^1(D_f, \R)$.
En første ordens differentialligning er autonom, hvis den opfylder følgende udtryk. 
%
\begin{align}
    \frac{dx}{dt}=f(x) \label{eq:def_autonome_første_ordens_differentialligning}
\end{align}
%
hvor funktionen $f$ ikke afhænger eksplicit af $t$. Hvis differentialligningen ikke er autonom, kaldes den for \textit{ikke-autonom}.
\end{defn}

%I dette projekt vil højere end første ordens differentialligning blive lavet om til et system af første ordens differentialligninger. Derfor vil definitionen af højere ordens autonome differentialligninger ikke blive gennemgået.

