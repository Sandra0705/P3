\section{djlkhlkjhlkhjgkjhgkgfkjhgæiejfo}
For en matrice i $\R^{2\times 2}$ er det muligt at bestemme udtrykket for $e^{tA}$ ved brug af egenværdierne og \textit{sporet} af matricen $A$. Udledningen af denne vil blive præsenteret i dette afsnit. Sporet er defineret som følgende 

\begin{minipage}\textwidth
\begin{defn}\textbf{}\label{def:spor} %Ny definition
\newline
Lad $A \in \R^{n\times n}$. Sporet af $A$ er defineret som 
\begin{align*}
    tr(A) = \sum_{i=1}^na_{ii}
\end{align*}
\end{defn}
\end{minipage}
%
Til nogle af de følgende resultater skal det karakteristiske polynomium anvendes. Det karakteristiske polynomium omskrives ved brug af sporet og determinanten. 
%
\begin{align}
    p_A(\lambda) &= det(A-\lambda I) \nonumber\\
    &=(a_{11}-\lambda)(a_{22}-\lambda)-(a_{12}\cdot a_{21}) \nonumber\\
    &=(a_{11}\cdot a_{22})-(a_{11}\cdot \lambda)-(a_{22}\cdot \lambda) + \lambda^2 - (a_{12}\cdot a_{21})\nonumber\\
    &=\lambda^2 - (a_{11} + a_{22})\lambda + (a_{11}\cdot a_{22} - a_{12}\cdot a_{21})\nonumber\\
    &= \lambda^2 - \left(tr(A)\right)\lambda + det(A) \label{eq:det_karakteristiske_polynomium_udtrykt_ved_sporet}
\intertext{Bemærk, at $\lambda$ er egenværdi til $A$, hvis $p_A(\lambda)=0$.}\nonumber
\end{align}

\begin{thmx} \textbf{Egenværdier for \textit{A} når $\bm{tr(A)=0}$} \label{sæt:egenværdier_og_spor}%Ny sætning
\newline
Lad $A\in \R^{2 \times 2}$ og $tr(A)=0$. Så gælder følgende for egenværdierne af $A$
\begin{enumerate}
    \item 0 er den eneste egenværdi med algebraisk multiplicitet 2, hvis $det(A)=0$
    \item Egenværdierne er givet ved $\pm\sqrt{-det(A)}$, hvis $det(A)<0$
    \item Egenværdierne er givet ved $\pm i\sqrt{det(A)}$, hvis $det(A)>0$
\end{enumerate}
\end{thmx}

\begin{bev} \textbf{} %Nyt bevis
\newline
Lad $A\in \R^{2 \times 2}$, og lad $tr(A)=0$. Da $tr(A)=0$, kan \eqref{eq:det_karakteristiske_polynomium_udtrykt_ved_sporet} udtrykkes som 
\begin{align} \label{eq:lambda^2+det(A)}
    \lambda^2+det(A)=0 \Leftrightarrow \lambda^2=-det(A)
\end{align}
\begin{itemize}
    \item [] \textbf{Bevis for punkt 1.}\\
    $det(A)=0$ indsættes i \eqref{eq:lambda^2+det(A)}.
    \begin{align*}
        \lambda^2&=-det(A)\\
        &=0
    \end{align*}
    Da $\lambda^2=0$, medfører det, at $\lambda_1=0$ og $\lambda_2=0$. Altså er 0 den eneste egenværdi med algebraisk multiplicitet 2. 
    \item [] \textbf{Bevis for punkt 2.}\\
    For $det(A) < 0$ gælder det, at 
    \begin{align*}
        \lambda^2 &= -det(A)\\
        &\Updownarrow\\
        \lambda &= \pm \sqrt{-det(A)}
    \end{align*}
    Denne omskrivning er mulig, da $det(A)<0 \Leftrightarrow -det(A)>0$. Altså gælder det for egenværdierne af $A$, at de er givet ved $\pm \sqrt{-det(A)}$, når $det(A)<0$.
    \item [] \textbf{Bevis for punkt 3.}\\
    For $det(A) > 0$ gælder det, at
    \begin{align*}
        \lambda^2 &= -det(A)\\
        &\Updownarrow\\
        \lambda &= \pm \sqrt{-det(A)}\\
        &= \pm\sqrt{-1}\sqrt{det(A)}\\
        &= \pm i\sqrt{det(A)}
    \end{align*}
    Altså gælder det for egenværdierne af $A$, at de er givet ved $\pm i\sqrt{det(A)}$, når $det(A)>0$.
\end{itemize}
\end{bev}

\autoref{sæt:egenværdier_og_spor} og Putzers algoritme vil nu blive anvendt til at bestemme udtrykket for $e^{tA}$.

\begin{thmx} \textbf{} \label{e^tA_udtrykt_når_trA=0}%Ny sætning
\newline
Lad $A\in \R^{2 \times 2}$ og $tr(A)=0$. Så gælder det, at hvis
\begin{enumerate}
    \item $det(A)=0$, så er $e^{tA}=I_2+tA$
    \item $det(A)<0$, så er $e^{tA}=cosh\left(t\sqrt{-det(A)}\right)I_2+\frac{sinh\left(t\sqrt{-det(A)}\right)}{\sqrt{-det(A)}}A$
    \item $det(A)>0$, så er $e^{tA}=cos\left(t\sqrt{det(A)}\right)I_2+\frac{sin\left(t\sqrt{det(A)}\right)}{\sqrt{det(A)}}A$
\end{enumerate}
\end{thmx}

\begin{bev} \textbf{} %Nyt bevis
\newline
I hvert punkt i dette bevis vil Putzers algoritme (\autoref{sæt:putzers_algoritme}) blive anvendt.
\begin{itemize}
    \item [] \textbf{Bevis for punkt 1.}\\
    For $det(A) = 0$ vides det fra \autoref{sæt:egenværdier_og_spor}, at egenværdierne af $A$ er 0 med multiplicitet 2. Altså
    \begin{align*}
        \lambda_1 = 0, \qquad \lambda_2 = 0
    \end{align*}
    Putzers algoritme anvendes til at bestemme udtrykket for $e^{tA}$. Først vil $r_1(t)$ og $r_2(t)$ blive beregnet.
    \begin{align*}
        r_1(t) &= e^{t\cdot \lambda_1} = e^{t \cdot 0} = 1\\
        r_2(t) &= e^{t\cdot \lambda_2} \cdot \int_0^te^{-s\cdot \lambda_2}r_1(s) ds = e^{t \cdot 0}\cdot \int_0^te^{-s\cdot 0}1 ds = 1\cdot \int_0^t1\cdot1 ds = t
    \intertext{Herefter vil $P_0$ og $P_1$ blive bestemt.}
        P_0 &= I_2\\
        P_1 &= (A-\lambda_1I_2) = (A - 0I_2) = A
    \intertext{Til sidst vil udtrykket for $e^{tA}$ blive bestemt.}
        e^{tA} &= \sum_{k=0}^1r_{k+1}(t)P_k = r_1 P_0 + r_2 P_1 = 1 I_2 + t A = I_2 + tA
    \end{align*}
    Altså gælder det, at $e^{tA} = I_2 + tA$ for $det(A)=0$.
    \item [] \textbf{Bevis for punkt 2.}\\
    For $det(A) < 0$ vides det fra \autoref{sæt:egenværdier_og_spor}, at egenværdierne af $A$ er givet ved $\pm \sqrt{-det(A)}$. Altså
    \begin{align*}
        \lambda_1 = \sqrt{-det(A)}, \qquad \lambda_2 = -\sqrt{-det(A)}
    \end{align*}
    Putzers algoritme anvendes til at bestemme udtrykket for $e^{tA}$. Først vil $r_1(t)$ og $r_2(t)$ blive bestemt.
    \begin{align*}
        r_1(t) &= e^{t\cdot \lambda_1} = e^{t \cdot \sqrt{-det(A)}}\\
        r_2(t) &=e^{t\cdot \lambda_2} \cdot \int_0^te^{-s\cdot \lambda_2}r_1(s) ds\\
        &=e^{t\cdot \left(-\sqrt{-det(A)}\right)} \cdot \int_0^te^{-s\cdot \left(-\sqrt{-det(A)}\right)}e^{s \cdot \sqrt{-det(A)}} ds
        \intertext{\autoref{sæt:egenskaber_for_eta} punkt 3 anvendes, og udtrykket integreres.}
        r_2(t)&= e^{-t\cdot \sqrt{-det(A)}}\cdot \int_0^te^{2s \cdot \sqrt{-det(A)}} ds\\
        &= e^{-t\cdot \sqrt{-det(A)}}\cdot \left[\frac{1}{2\cdot\sqrt{-det(A)}} \cdot e^{2s\cdot \sqrt{-det(A)}}\right]_0^t\\
        &= e^{-t\cdot \sqrt{-det(A)}}\cdot \left(\frac{1}{2\cdot\sqrt{-det(A)}}\cdot e^{2t\cdot \sqrt{-det(A)}} - \frac{1}{2\cdot\sqrt{-det(A)}}\right)\\
        &= \frac{1}{2 \cdot \sqrt{-det(A)}}\cdot e^{t\cdot\sqrt{-det(A)}} - \frac{1}{2 \cdot \sqrt{-det(A)}}\cdot e^{-t\cdot\sqrt{-det(A)}}\\
        &=\frac{e^{t\cdot\sqrt{-det(A)}}-e^{-t\cdot\sqrt{-det(A)}}}{2 \sqrt{-det(A)}}
        \intertext{Ved at anvende \autoref{bilag:cos_og_sin} fås følgende}
        r_2(t)&=\frac{sinh\left(t\cdot \sqrt{-det(A)}\right)}{\sqrt{-det(A)}}
        \intertext{Herefter vil $P_0$ og $P_1$ blive bestemt.}
        P_0 &= I_2\\
        P_1 &= (A-\lambda_1I_2) = \left(A - \sqrt{-det(A)}I_2\right) 
        \intertext{Til sidst vil udtrykket for $e^{tA}$ blive bestemt.}
        e^{tA} &= \sum_{k=0}^1r_{k+1}(t)P_k = r_1P_0 + r_2P_1 \\
        &= e^{t \cdot \sqrt{-det(A)}}I_2 + \frac{sinh\left(t\cdot \sqrt{-det(A)}\right)}{\sqrt{-det(A)}}\left(A - \sqrt{-det(A)}I_2\right)  \\
        &= e^{t \cdot \sqrt{-det(A)}}I_2 + \frac{sinh\left(t\cdot \sqrt{-det(A)}\right)}{\sqrt{-det(A)}} A - {sinh\left(t\cdot \sqrt{-det(A)}\right)}I_2\\
        \intertext{Regnereglerne for hyperbolsk cosinus og sinus anvendes, se \autoref{bilag:cos_og_sin}}
        &= e^{t \cdot \sqrt{-det(A)}}  I_2 + \frac{sinh\left(t\cdot \sqrt{-det(A)}\right)}{\sqrt{-det(A)}}A - \frac{e^{t\cdot\sqrt{-det(A)}}-e^{-t\cdot\sqrt{-det(A)}}}{2} I_2\\
        &= \left(e^{t \cdot \sqrt{-det(A)}} - \frac{e^{t\cdot\sqrt{-det(A)}}}{2} + \frac{e^{-t\cdot\sqrt{-det(A)}}}{2} \right) I_2 + \frac{sinh\left(t\cdot \sqrt{-det(A)}\right)}{\sqrt{-det(A)}}A\\
        &= \frac{e^{t \cdot \sqrt{-det(A)}}+e^{-t \cdot \sqrt{-det(A)}}}{2}I_2+ \frac{sinh\left(t\cdot \sqrt{-det(A)}\right)}{\sqrt{-det(A)}}A\\
        &= cosh\left(t\cdot \sqrt{-det(A)}\right) I_2 + \frac{sinh\left(t\cdot \sqrt{-det(A)}\right)}{\sqrt{-det(A)}} A
    \end{align*}
    Altså gælder det, at $e^{tA} = cosh\left(t\cdot \sqrt{-det(A)}\right) I_2 + \frac{sinh\left(t\cdot \sqrt{-det(A)}\right)}{\sqrt{-det(A)}}A$ for $det(A)<0$.
    \item [] \textbf{Bevis for punkt 3.}\\    
    For $det(A)>0$ vides det fra \autoref{sæt:egenværdier_og_spor}, at egenværdierne af $A$ er givet ved $\pm i\sqrt{det(A)}$. Altså
    \begin{align*}
        \lambda_1=i\sqrt{det(A)},\qquad \lambda_2=-i\sqrt{det(A)}
    \end{align*}   
    Putzers algoritme anvendes til at bestemme udtrykket for $e^{tA}$. Først vil $r_1(t)$ og $r_2(t)$ blive beregnet.
    \begin{align*}
        r_1(t) &= e^{t\cdot \lambda_1} = e^{t \cdot i\sqrt{det(A)}} \\
        r_2(t) &= e^{t\cdot \lambda_2} \cdot \int_0^te^{-s\cdot \lambda_2}r_1(s) ds\\
        &= e^{t\cdot -i\sqrt{det(A)}} \cdot \int_0^te^{-s\cdot \left(-i\sqrt{det(A)}\right)} e^{s \cdot i\sqrt{det(A)}} ds\\
        \intertext{\autoref{sæt:egenskaber_for_eta} punkt 3 anvendes, og udtrykket integreres.}
         r_2(t)&=e^{t\cdot -i\sqrt{det(A)}}\int_0^s e^{2si\cdot \sqrt{det(A)}}\\
         &=e^{-t\cdot i\sqrt{det(A)}}\cdot \left[\frac{1}{2\cdot i\sqrt{det(A)}}e^{2s\cdot i\sqrt{det(A)}}\right]^t_0\\
         &= e^{-t\cdot i\sqrt{det(A)}}\cdot \left(\frac{1}{2\cdot i\sqrt{det(A)}}\cdot e^{2t\cdot i\sqrt{det(A)}} - \frac{1}{2\cdot i\sqrt{det(A)}}\right)\\
         &= \frac{1}{2 \cdot i\sqrt{det(A)}}\cdot e^{t\cdot i\sqrt{det(A)}} - \frac{1}{2 \cdot i\sqrt{det(A)}}\cdot e^{-t\cdot i\sqrt{det(A)}}\\
         &=\frac{1}{\sqrt{det(A)}} \cdot \frac{e^{t\cdot i\sqrt{det(A)}}-e^{-t\cdot i\sqrt{det(A)}}}{2\cdot i}
         \intertext{Ved at anvende \autoref{bilag:cos_og_sin} fås følgende}
         r_2(t)&=\frac{1}{\sqrt{det(A)}} \cdot sin\left(t\sqrt{det(A)}\right)\\
         &= \frac{sin\left(t\sqrt{det(A)}\right)}{\sqrt{det(A)}}
          \intertext{Herefter vil $P_0$ og $P_1$ blive bestemt.}
        P_0 &= I_2\\
        P_1 &= (A-\lambda_1 I_2) = \left(A - \left(i\sqrt{det(A)}\right)I_2\right) 
        \intertext{Tilsidst vil udtrykket for $e^{tA}$ blive bestemt.}
        e^{tA} &= \sum_{k=0}^1r_{k+1}(t)P_k = r_1 P_0 + r_2 P_1 \\
        &= e^{t\cdot i\sqrt{det(A)}} I_2 + \frac{sin\left(t\sqrt{det(A)}\right)}{\sqrt{det(A)}} \left(A - \left(i\sqrt{det(A)}\right) I_2\right)\\
        &= e^{t\cdot i\sqrt{det(A)}}I_2 + \frac{sin\left(t\sqrt{det(A)}\right)}{\sqrt{det(A)}} A - sin\left(t\sqrt{det(A)}\right) i I_2\\
        \intertext{Regnereglerne for cosinus og sinus anvendes, \eqref{eq:sin_og_cos}}.
        e^{tA}&= e^{t\cdot i\sqrt{det(A)}}I_2 + \frac{sin\left(t\sqrt{det(A)}\right)}{\sqrt{det(A)}}A - \left(\frac{e^{t\cdot i\sqrt{det(A)}}-e^{-t\cdot i\sqrt{det(A)}}}{2\cdot i}\right) iI_2\\
        &= e^{t\cdot i\sqrt{det(A)}}I_2 + \frac{sin\left(t\sqrt{det(A)}\right)}{\sqrt{det(A)}}A - \frac{1}{2}\left({e^{t\cdot i\sqrt{det(A)}}-e^{-t\cdot i\sqrt{det(A)}}}\right)I_2\\
        &= \left(e^{t\cdot i\sqrt{det(A)}} - \frac{1}{2}e^{t\cdot i \sqrt{det(A)}} + \frac{1}{2}e^{-t\cdot i \sqrt{det(A)}}\right) I_2 + \frac{sin\left(t\sqrt{det(A)}\right)}{\sqrt{det(A)}} A\\
        &= \frac{e^{t\cdot i\sqrt{det(A)}} - e^{-t\cdot i\sqrt{det(A)}}}{2} I_2 + \frac{sin\left(t\sqrt{det(A)}\right)}{\sqrt{det(A)}} A\\
        &= cos\left(t\sqrt{det(A)}\right) I_2 + \frac{sin\left(t\sqrt{det(A)}\right)}{\sqrt{det(A)}} A
    \end{align*}
    Altså gælder det, at $e^{tA} = cos\left(t\cdot \sqrt{det(A)}\right) I_2+ \frac{sin\left(t\cdot \sqrt{det(A)}\right)}{\sqrt{det(A)}} A$ for $det(A)>0$.
\end{itemize}
\end{bev}

Selvom egenværdierne i \autoref{e^tA_udtrykt_når_trA=0} punkt 3 er komplekse, så er udtrykket for $e^{tA}$ i punkt 3 reelt. For en reel matrice, $A$, gælder der, at $e^{tA}$ er en reel matrice. Udtrykkene for $e^{tA}$ fra \autoref{e^tA_udtrykt_når_trA=0} gælder, når $tr(A)=0$. Der vil nu blive gennemgået nogle resultater, der bruges til udledningen af udtrykket for $e^{tA}$, når $tr(A) \neq 0$. 

