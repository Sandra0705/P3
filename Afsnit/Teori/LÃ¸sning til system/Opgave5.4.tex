\begin{thmx}\textbf{Regneregler} \label{sæt:regneregler_for_etcI}
\newline
Lad $c\in\R$ og $A\in\R^{2\times2}$. Da gælder følgende
\begin{enumerate}
    \item $e^{tcI_2} = e^{tc}I_2$
    \item $e^{t(A+cI_2)} = e^{tc}e^{tA}$
\end{enumerate}
\end{thmx}
%
\begin{bev}\textbf{}
\begin{itemize}
\item [] \textbf{Bevis for punkt 1}\\
Punkt 1 vil blive bevist ud fra Putzers algoritme (se \textbf{Ref Sætning omkring putzers}). \textbf{Lav evt en generel ref op til hvordan man beregner putzer og skriv at denne vil blive brugt i dette bevis}
Lad $A=I_2$. Identitetsmatricen $I_2$ har egenværdierne $\lambda_1,\lambda_2=1$.

Først bestemmes $r_1(t)$ og $r_2(t)$. 
%
\begin{align*}
    r_1 &= e^{tcI_2} = e^{tc} \\
    r_2 &= e^{tcI_2}\cdot \int_0^te^{-sc}e^{sc} ds
\end{align*}
%
Herefter opstilles $P_0$ og $P_1$.
\begin{align*}
    P_0 = I_2
    P_1 &= (I_2 - 1I_2) \\
    &= (I_2-I_2)\\
    &= 0 
\end{align*}
%
\textbf{ref} benyttes til at udregne $e^{tcI_2}$. 
%
\begin{align*}
    e^{tcI_2} &= \sum_{k=0}^1 r_{k+1}P_k \\
    &= r_1P_0 + r_2P_1\\
    &= e^{tc} \cdot I_2 + \left(e^{tcI_2}\cdot \int_0^te^{-sc}e^{sc}\right) \cdot 0 \\
    &= e^{tc}I_2
\end{align*}
Dermed er det bevist, at $e^{tcI_2} = e^{tc}I_2$.

\item [] \textbf{Bevis for punkt 2}\\
Punkt 2 bevises med hjælp fra \autoref{sæt:egenskaber_for_eta}.
\begin{align*}
    e^{t(A+cI_2)} &= e^{tA+tcI_2} \\ 
    &= e^{tcI_2+tA}\\ 
    &=e^{tcI_2}e^{tA} \\
    &= e^{tc}e^{tA}
\end{align*}
%
Dermed er det bevist, at $e^{t(A+cI_2)}=e^{tc}e^{tA}$.
%
\end{itemize}
\end{bev}