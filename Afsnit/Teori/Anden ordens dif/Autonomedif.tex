\section{Autonome lineære systemer}
På samme måde som en funktion kan være autonom, kan et system også være autonomt.

\begin{minipage}\textwidth
\begin{defn}\textbf{Autonomt lineært system} %Ny definition
\newline
Lad $t \in I$ og $\textbf{x}: I \to \R^n$.
Lad $D_\textbf{f} \subseteq \R^n$ være en åben ikke-tom delmængde, og $\textbf{f}\in C^1(D_\textbf{f},\R^n)$. $\textbf{f}$ er et autonomt linært system, hvis
\begin{align*}
    \frac{d\textbf{x}}{dt} = \textbf{f}(\textbf{x})
\end{align*}
Altså afhænger systemet ikke eksplicit af $t$. Hvis systemet ikke er autonomt, kaldes det ikke-autonomt. 
\end{defn}
\end{minipage}




