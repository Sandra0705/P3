\section{Anden ordens differentialligninger som system}

I dette afsnit vil differentialligninger af anden orden blive introduceret. For en anden ordens differentialligning gælder det, at ordenen af den højeste afledede er én. En anden ordens differentialligning er defineret ved følgende. 
\begin{defn}\textbf{Anden ordens differentialligning} %Ny definition
\newline
En anden orden differentialligning med de ukendte funktioner $x, \frac{dx}{dt}$ er 
\begin{align*}
    \frac{d^2x}{dt^2} = f\left(t, x, \frac{dx}{dt}\right)
\end{align*}
hvor $f$ er givet. En anden ordens differentialligning kan skrives på formen
\begin{align*}
    \frac{d^2x}{dt^2}=a_1(t)\frac{dx}{dt}+a_0(t)x + b(t)
\end{align*}
Den lineære differentialligning har konstante koefficienter, hvis $a_1(t)=a_1$, $a_0(t)=a_0$ og $b(t)=b$. Ellers har den lineære differentialligning variable koefficienter.
\end{defn}

En anden ordens differentialligning kan omskrives til et system af første ordens differentialligninger på følgende måde. Først betragtes en anden ordens differentialligning. 
\begin{align}\label{eq:anden_ordens_diff}
    \frac{d^2x}{dt^2}= a_1(t)\frac{dx}{dt}+a_0(t)x + b(t)
\end{align}
%
Ud fra ovenstående defineres følgende funktioner.
%
\begin{align*}
    x_1&=x\\
    x_2 &= \frac{dx}{dt}
\end{align*}
%
Ved brug af ovenstående funktioner omskrives \eqref{eq:anden_ordens_diff} til følgende system af første ordens differentialligninger.
%
\begin{align*}
    \frac{dx_1}{dt} &= x_2\\
    \frac{dx_2}{dt} &= a_1(t)x_2 + a_0(t)x_1 + b(t)
\end{align*}
%
Dette kan også udtrykkes som et matricevektor-produkt.
%
\begin{align*}
    \begin{bmatrix}
        \frac{dx_1}{dt}\\
        \frac{dx_2}{dt}
    \end{bmatrix}
    =
    \begin{bmatrix}
    0 & 1\\
    a_0 & a_1
    \end{bmatrix}
    \begin{bmatrix}
        x_1(t)\\
        x_2(t)
    \end{bmatrix}
    + 
    \begin{bmatrix}
        0\\
        b(t)
    \end{bmatrix}
\end{align*}

Hermed er en anden ordens differentialligning omskrevet til et system af første ordens differentialligninger.

\section{Autonome lineære systemer}
På samme måde som en funktion kan være autonom, kan et system også være autonomt.

\begin{minipage}\textwidth
\begin{defn}\textbf{Autonomt lineært system} %Ny definition
\newline
Lad $D_\textbf{f} \subseteq \R^n$ være en åben ikke-tom delmængde og $\textbf{f}\in C^0(D_\textbf{f},\R^n)$. Systemet af differentialligninger er et autonomt lineært system, hvis
\begin{align*}
    \frac{d\textbf{x}}{dt} = \textbf{f}(\textbf{x})
\end{align*}
hvor $t \in I$ og $\textbf{x}\in C^1(I, D_\textbf{f})$. Altså afhænger systemet ikke eksplicit af $t$. Hvis systemet ikke er autonomt, kaldes det ikke-autonomt. 
\end{defn}
\end{minipage}




