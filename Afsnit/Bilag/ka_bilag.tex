\section{Bilag til KA}
\subsubsection{Ortogonal dekomposition}
Denne sætning og det tilhørende bevis er baseret på \cite[s. 170-171]{"LAMA"}.

\begin{thmx} \textbf{Ortogonal dekomposition} \label{sæt:ortogonal_dekomposition}
\newline
Antag $\textbf{u},\textbf{v} \in \R^n$ og $\textbf{v} \neq \textbf{0}$. Lad $c=\displaystyle\frac{\langle \textbf{u} , \textbf{v} \rangle}{\norm{\textbf{v}}^2}$ og $\textbf{w}=\textbf{u}-\displaystyle\frac{\langle \textbf{u} , \textbf{v}\rangle}{\norm{\textbf{v}}^2}\textbf{v}$. \\
Da vil $\langle\textbf{w}, \textbf{v}\rangle = 0$ og $\textbf{u}=c\textbf{v}-\textbf{w}$.
\end{thmx}

\begin{bev}\textbf{} 
\newline
Antag $\textbf{u},\textbf{v} \in \R^n$, $c \in \R$ og $\textbf{v} \neq \textbf{0}$. Der skal bestemmes et udtryk, hvor \textbf{u} kan opskrives som et multiplum af \textbf{v} adderet med en ortogonal vektor til \textbf{v}, som kaldes \textbf{w}. Dette kan også skrives som
\begin{align}
    \textbf{u} &= c\textbf{v} + \textbf{w} \nonumber \\
    &= c\textbf{v} + (\textbf{u}-c\textbf{v}) \label{eq:ortogonal_bevis_indsæt_c}
\intertext{Dernæst skal der bestemmes en skalar, $c$, således, at \textbf{v} er ortogonal på $(\textbf{u}-c\textbf{v})$.}
    0 &= \langle\textbf{u}-c\textbf{v},\textbf{v}\rangle \nonumber\\
    &= \langle\textbf{u},\textbf{v}\rangle - c\norm{\textbf{v}}^2 \label{eq:ortogonal_bevis_vælg_c}
\intertext{Ud fra \eqref{eq:ortogonal_bevis_vælg_c} skal der vælges en skalar, $c$, således, at $c=\displaystyle\frac{\langle\textbf{u},\textbf{v}\rangle}{\norm{\textbf{v}}^2}$. Skalaren indsættes i \eqref{eq:ortogonal_bevis_indsæt_c}.}
    \textbf{u}&=\frac{\langle\textbf{u},\textbf{v}\rangle}{\norm{\textbf{v}}^2}\textbf{v}+\left(\textbf{u}-\frac{\langle\textbf{u},\textbf{v}\rangle}{\norm{\textbf{v}}^2}\textbf{v}\right) \nonumber
\end{align}
Dermed er \autoref{sæt:ortogonal_dekomposition} bevist.
\end{bev}

\subsubsection{Pythagoras' sætning}
Denne sætning og tilhørende bevis er baseret på \cite[s. 170]{"LAMA"}.
\begin{thmx} \textbf{Pythogaros' sætning}
\label{sæt:pythagoras_sætning}
\newline
Antag \textbf{u} og \textbf{v} er ortogonale vektorer i $\R^n$. Da vil der gælde, at
\begin{align*}
    \norm{\textbf{u}+\textbf{v}}^2 =
    \norm{\textbf{u}}^2+\norm{\textbf{v}}^2
\end{align*}
\end{thmx}
%
\begin{bev} \textbf{}
\newline    
Det vides, at
\begin{align*}
    \norm{\textbf{u}+\textbf{v}}^2 &= \langle\textbf{u+v},\textbf{u+v}\rangle \\
    &= \langle\textbf{u},\textbf{u}\rangle+\langle\textbf{u},\textbf{v}\rangle + \langle\textbf{v},\textbf{u}\rangle + \langle\textbf{v},\textbf{v}\rangle \\
    &= \norm{\textbf{u}}^2 + \norm{\textbf{v}}^2
\end{align*}
Derved er \autoref{sæt:pythagoras_sætning} bevist.
\end{bev}